\documentclass[11pt]{article}

    \usepackage[breakable]{tcolorbox}
    \usepackage{parskip} % Stop auto-indenting (to mimic markdown behaviour)
    
    \usepackage{iftex}
    \ifPDFTeX
    	\usepackage[T1]{fontenc}
    	\usepackage{mathpazo}
    \else
    	\usepackage{fontspec}
    \fi

    % Basic figure setup, for now with no caption control since it's done
    % automatically by Pandoc (which extracts ![](path) syntax from Markdown).
    \usepackage{graphicx}
    % Maintain compatibility with old templates. Remove in nbconvert 6.0
    \let\Oldincludegraphics\includegraphics
    % Ensure that by default, figures have no caption (until we provide a
    % proper Figure object with a Caption API and a way to capture that
    % in the conversion process - todo).
    \usepackage{caption}
    \DeclareCaptionFormat{nocaption}{}
    \captionsetup{format=nocaption,aboveskip=0pt,belowskip=0pt}

    \usepackage[Export]{adjustbox} % Used to constrain images to a maximum size
    \adjustboxset{max size={0.9\linewidth}{0.9\paperheight}}
    \usepackage{float}
    \floatplacement{figure}{H} % forces figures to be placed at the correct location
    \usepackage{xcolor} % Allow colors to be defined
    \usepackage{enumerate} % Needed for markdown enumerations to work
    \usepackage{geometry} % Used to adjust the document margins
    \usepackage{amsmath} % Equations
    \usepackage{amssymb} % Equations
    \usepackage{textcomp} % defines textquotesingle
    % Hack from http://tex.stackexchange.com/a/47451/13684:
    \AtBeginDocument{%
        \def\PYZsq{\textquotesingle}% Upright quotes in Pygmentized code
    }
    \usepackage{upquote} % Upright quotes for verbatim code
    \usepackage{eurosym} % defines \euro
    \usepackage[mathletters]{ucs} % Extended unicode (utf-8) support
    \usepackage{fancyvrb} % verbatim replacement that allows latex
    \usepackage{grffile} % extends the file name processing of package graphics 
                         % to support a larger range
    \makeatletter % fix for grffile with XeLaTeX
    \def\Gread@@xetex#1{%
      \IfFileExists{"\Gin@base".bb}%
      {\Gread@eps{\Gin@base.bb}}%
      {\Gread@@xetex@aux#1}%
    }
    \makeatother

    % The hyperref package gives us a pdf with properly built
    % internal navigation ('pdf bookmarks' for the table of contents,
    % internal cross-reference links, web links for URLs, etc.)
    \usepackage{hyperref}
    % The default LaTeX title has an obnoxious amount of whitespace. By default,
    % titling removes some of it. It also provides customization options.
    \usepackage{titling}
    \usepackage{longtable} % longtable support required by pandoc >1.10
    \usepackage{booktabs}  % table support for pandoc > 1.12.2
    \usepackage[inline]{enumitem} % IRkernel/repr support (it uses the enumerate* environment)
    \usepackage[normalem]{ulem} % ulem is needed to support strikethroughs (\sout)
                                % normalem makes italics be italics, not underlines
    \usepackage{mathrsfs}
    

    
    % Colors for the hyperref package
    \definecolor{urlcolor}{rgb}{0,.145,.698}
    \definecolor{linkcolor}{rgb}{.71,0.21,0.01}
    \definecolor{citecolor}{rgb}{.12,.54,.11}

    % ANSI colors
    \definecolor{ansi-black}{HTML}{3E424D}
    \definecolor{ansi-black-intense}{HTML}{282C36}
    \definecolor{ansi-red}{HTML}{E75C58}
    \definecolor{ansi-red-intense}{HTML}{B22B31}
    \definecolor{ansi-green}{HTML}{00A250}
    \definecolor{ansi-green-intense}{HTML}{007427}
    \definecolor{ansi-yellow}{HTML}{DDB62B}
    \definecolor{ansi-yellow-intense}{HTML}{B27D12}
    \definecolor{ansi-blue}{HTML}{208FFB}
    \definecolor{ansi-blue-intense}{HTML}{0065CA}
    \definecolor{ansi-magenta}{HTML}{D160C4}
    \definecolor{ansi-magenta-intense}{HTML}{A03196}
    \definecolor{ansi-cyan}{HTML}{60C6C8}
    \definecolor{ansi-cyan-intense}{HTML}{258F8F}
    \definecolor{ansi-white}{HTML}{C5C1B4}
    \definecolor{ansi-white-intense}{HTML}{A1A6B2}
    \definecolor{ansi-default-inverse-fg}{HTML}{FFFFFF}
    \definecolor{ansi-default-inverse-bg}{HTML}{000000}

    % commands and environments needed by pandoc snippets
    % extracted from the output of `pandoc -s`
    \providecommand{\tightlist}{%
      \setlength{\itemsep}{0pt}\setlength{\parskip}{0pt}}
    \DefineVerbatimEnvironment{Highlighting}{Verbatim}{commandchars=\\\{\}}
    % Add ',fontsize=\small' for more characters per line
    \newenvironment{Shaded}{}{}
    \newcommand{\KeywordTok}[1]{\textcolor[rgb]{0.00,0.44,0.13}{\textbf{{#1}}}}
    \newcommand{\DataTypeTok}[1]{\textcolor[rgb]{0.56,0.13,0.00}{{#1}}}
    \newcommand{\DecValTok}[1]{\textcolor[rgb]{0.25,0.63,0.44}{{#1}}}
    \newcommand{\BaseNTok}[1]{\textcolor[rgb]{0.25,0.63,0.44}{{#1}}}
    \newcommand{\FloatTok}[1]{\textcolor[rgb]{0.25,0.63,0.44}{{#1}}}
    \newcommand{\CharTok}[1]{\textcolor[rgb]{0.25,0.44,0.63}{{#1}}}
    \newcommand{\StringTok}[1]{\textcolor[rgb]{0.25,0.44,0.63}{{#1}}}
    \newcommand{\CommentTok}[1]{\textcolor[rgb]{0.38,0.63,0.69}{\textit{{#1}}}}
    \newcommand{\OtherTok}[1]{\textcolor[rgb]{0.00,0.44,0.13}{{#1}}}
    \newcommand{\AlertTok}[1]{\textcolor[rgb]{1.00,0.00,0.00}{\textbf{{#1}}}}
    \newcommand{\FunctionTok}[1]{\textcolor[rgb]{0.02,0.16,0.49}{{#1}}}
    \newcommand{\RegionMarkerTok}[1]{{#1}}
    \newcommand{\ErrorTok}[1]{\textcolor[rgb]{1.00,0.00,0.00}{\textbf{{#1}}}}
    \newcommand{\NormalTok}[1]{{#1}}
    
    % Additional commands for more recent versions of Pandoc
    \newcommand{\ConstantTok}[1]{\textcolor[rgb]{0.53,0.00,0.00}{{#1}}}
    \newcommand{\SpecialCharTok}[1]{\textcolor[rgb]{0.25,0.44,0.63}{{#1}}}
    \newcommand{\VerbatimStringTok}[1]{\textcolor[rgb]{0.25,0.44,0.63}{{#1}}}
    \newcommand{\SpecialStringTok}[1]{\textcolor[rgb]{0.73,0.40,0.53}{{#1}}}
    \newcommand{\ImportTok}[1]{{#1}}
    \newcommand{\DocumentationTok}[1]{\textcolor[rgb]{0.73,0.13,0.13}{\textit{{#1}}}}
    \newcommand{\AnnotationTok}[1]{\textcolor[rgb]{0.38,0.63,0.69}{\textbf{\textit{{#1}}}}}
    \newcommand{\CommentVarTok}[1]{\textcolor[rgb]{0.38,0.63,0.69}{\textbf{\textit{{#1}}}}}
    \newcommand{\VariableTok}[1]{\textcolor[rgb]{0.10,0.09,0.49}{{#1}}}
    \newcommand{\ControlFlowTok}[1]{\textcolor[rgb]{0.00,0.44,0.13}{\textbf{{#1}}}}
    \newcommand{\OperatorTok}[1]{\textcolor[rgb]{0.40,0.40,0.40}{{#1}}}
    \newcommand{\BuiltInTok}[1]{{#1}}
    \newcommand{\ExtensionTok}[1]{{#1}}
    \newcommand{\PreprocessorTok}[1]{\textcolor[rgb]{0.74,0.48,0.00}{{#1}}}
    \newcommand{\AttributeTok}[1]{\textcolor[rgb]{0.49,0.56,0.16}{{#1}}}
    \newcommand{\InformationTok}[1]{\textcolor[rgb]{0.38,0.63,0.69}{\textbf{\textit{{#1}}}}}
    \newcommand{\WarningTok}[1]{\textcolor[rgb]{0.38,0.63,0.69}{\textbf{\textit{{#1}}}}}
    
    
    % Define a nice break command that doesn't care if a line doesn't already
    % exist.
    \def\br{\hspace*{\fill} \\* }
    % Math Jax compatibility definitions
    \def\gt{>}
    \def\lt{<}
    \let\Oldtex\TeX
    \let\Oldlatex\LaTeX
    \renewcommand{\TeX}{\textrm{\Oldtex}}
    \renewcommand{\LaTeX}{\textrm{\Oldlatex}}
    % Document parameters
    % Document title
    \title{bunshun\_online\_1}
    
    
    
    
    
% Pygments definitions
\makeatletter
\def\PY@reset{\let\PY@it=\relax \let\PY@bf=\relax%
    \let\PY@ul=\relax \let\PY@tc=\relax%
    \let\PY@bc=\relax \let\PY@ff=\relax}
\def\PY@tok#1{\csname PY@tok@#1\endcsname}
\def\PY@toks#1+{\ifx\relax#1\empty\else%
    \PY@tok{#1}\expandafter\PY@toks\fi}
\def\PY@do#1{\PY@bc{\PY@tc{\PY@ul{%
    \PY@it{\PY@bf{\PY@ff{#1}}}}}}}
\def\PY#1#2{\PY@reset\PY@toks#1+\relax+\PY@do{#2}}

\expandafter\def\csname PY@tok@w\endcsname{\def\PY@tc##1{\textcolor[rgb]{0.73,0.73,0.73}{##1}}}
\expandafter\def\csname PY@tok@c\endcsname{\let\PY@it=\textit\def\PY@tc##1{\textcolor[rgb]{0.25,0.50,0.50}{##1}}}
\expandafter\def\csname PY@tok@cp\endcsname{\def\PY@tc##1{\textcolor[rgb]{0.74,0.48,0.00}{##1}}}
\expandafter\def\csname PY@tok@k\endcsname{\let\PY@bf=\textbf\def\PY@tc##1{\textcolor[rgb]{0.00,0.50,0.00}{##1}}}
\expandafter\def\csname PY@tok@kp\endcsname{\def\PY@tc##1{\textcolor[rgb]{0.00,0.50,0.00}{##1}}}
\expandafter\def\csname PY@tok@kt\endcsname{\def\PY@tc##1{\textcolor[rgb]{0.69,0.00,0.25}{##1}}}
\expandafter\def\csname PY@tok@o\endcsname{\def\PY@tc##1{\textcolor[rgb]{0.40,0.40,0.40}{##1}}}
\expandafter\def\csname PY@tok@ow\endcsname{\let\PY@bf=\textbf\def\PY@tc##1{\textcolor[rgb]{0.67,0.13,1.00}{##1}}}
\expandafter\def\csname PY@tok@nb\endcsname{\def\PY@tc##1{\textcolor[rgb]{0.00,0.50,0.00}{##1}}}
\expandafter\def\csname PY@tok@nf\endcsname{\def\PY@tc##1{\textcolor[rgb]{0.00,0.00,1.00}{##1}}}
\expandafter\def\csname PY@tok@nc\endcsname{\let\PY@bf=\textbf\def\PY@tc##1{\textcolor[rgb]{0.00,0.00,1.00}{##1}}}
\expandafter\def\csname PY@tok@nn\endcsname{\let\PY@bf=\textbf\def\PY@tc##1{\textcolor[rgb]{0.00,0.00,1.00}{##1}}}
\expandafter\def\csname PY@tok@ne\endcsname{\let\PY@bf=\textbf\def\PY@tc##1{\textcolor[rgb]{0.82,0.25,0.23}{##1}}}
\expandafter\def\csname PY@tok@nv\endcsname{\def\PY@tc##1{\textcolor[rgb]{0.10,0.09,0.49}{##1}}}
\expandafter\def\csname PY@tok@no\endcsname{\def\PY@tc##1{\textcolor[rgb]{0.53,0.00,0.00}{##1}}}
\expandafter\def\csname PY@tok@nl\endcsname{\def\PY@tc##1{\textcolor[rgb]{0.63,0.63,0.00}{##1}}}
\expandafter\def\csname PY@tok@ni\endcsname{\let\PY@bf=\textbf\def\PY@tc##1{\textcolor[rgb]{0.60,0.60,0.60}{##1}}}
\expandafter\def\csname PY@tok@na\endcsname{\def\PY@tc##1{\textcolor[rgb]{0.49,0.56,0.16}{##1}}}
\expandafter\def\csname PY@tok@nt\endcsname{\let\PY@bf=\textbf\def\PY@tc##1{\textcolor[rgb]{0.00,0.50,0.00}{##1}}}
\expandafter\def\csname PY@tok@nd\endcsname{\def\PY@tc##1{\textcolor[rgb]{0.67,0.13,1.00}{##1}}}
\expandafter\def\csname PY@tok@s\endcsname{\def\PY@tc##1{\textcolor[rgb]{0.73,0.13,0.13}{##1}}}
\expandafter\def\csname PY@tok@sd\endcsname{\let\PY@it=\textit\def\PY@tc##1{\textcolor[rgb]{0.73,0.13,0.13}{##1}}}
\expandafter\def\csname PY@tok@si\endcsname{\let\PY@bf=\textbf\def\PY@tc##1{\textcolor[rgb]{0.73,0.40,0.53}{##1}}}
\expandafter\def\csname PY@tok@se\endcsname{\let\PY@bf=\textbf\def\PY@tc##1{\textcolor[rgb]{0.73,0.40,0.13}{##1}}}
\expandafter\def\csname PY@tok@sr\endcsname{\def\PY@tc##1{\textcolor[rgb]{0.73,0.40,0.53}{##1}}}
\expandafter\def\csname PY@tok@ss\endcsname{\def\PY@tc##1{\textcolor[rgb]{0.10,0.09,0.49}{##1}}}
\expandafter\def\csname PY@tok@sx\endcsname{\def\PY@tc##1{\textcolor[rgb]{0.00,0.50,0.00}{##1}}}
\expandafter\def\csname PY@tok@m\endcsname{\def\PY@tc##1{\textcolor[rgb]{0.40,0.40,0.40}{##1}}}
\expandafter\def\csname PY@tok@gh\endcsname{\let\PY@bf=\textbf\def\PY@tc##1{\textcolor[rgb]{0.00,0.00,0.50}{##1}}}
\expandafter\def\csname PY@tok@gu\endcsname{\let\PY@bf=\textbf\def\PY@tc##1{\textcolor[rgb]{0.50,0.00,0.50}{##1}}}
\expandafter\def\csname PY@tok@gd\endcsname{\def\PY@tc##1{\textcolor[rgb]{0.63,0.00,0.00}{##1}}}
\expandafter\def\csname PY@tok@gi\endcsname{\def\PY@tc##1{\textcolor[rgb]{0.00,0.63,0.00}{##1}}}
\expandafter\def\csname PY@tok@gr\endcsname{\def\PY@tc##1{\textcolor[rgb]{1.00,0.00,0.00}{##1}}}
\expandafter\def\csname PY@tok@ge\endcsname{\let\PY@it=\textit}
\expandafter\def\csname PY@tok@gs\endcsname{\let\PY@bf=\textbf}
\expandafter\def\csname PY@tok@gp\endcsname{\let\PY@bf=\textbf\def\PY@tc##1{\textcolor[rgb]{0.00,0.00,0.50}{##1}}}
\expandafter\def\csname PY@tok@go\endcsname{\def\PY@tc##1{\textcolor[rgb]{0.53,0.53,0.53}{##1}}}
\expandafter\def\csname PY@tok@gt\endcsname{\def\PY@tc##1{\textcolor[rgb]{0.00,0.27,0.87}{##1}}}
\expandafter\def\csname PY@tok@err\endcsname{\def\PY@bc##1{\setlength{\fboxsep}{0pt}\fcolorbox[rgb]{1.00,0.00,0.00}{1,1,1}{\strut ##1}}}
\expandafter\def\csname PY@tok@kc\endcsname{\let\PY@bf=\textbf\def\PY@tc##1{\textcolor[rgb]{0.00,0.50,0.00}{##1}}}
\expandafter\def\csname PY@tok@kd\endcsname{\let\PY@bf=\textbf\def\PY@tc##1{\textcolor[rgb]{0.00,0.50,0.00}{##1}}}
\expandafter\def\csname PY@tok@kn\endcsname{\let\PY@bf=\textbf\def\PY@tc##1{\textcolor[rgb]{0.00,0.50,0.00}{##1}}}
\expandafter\def\csname PY@tok@kr\endcsname{\let\PY@bf=\textbf\def\PY@tc##1{\textcolor[rgb]{0.00,0.50,0.00}{##1}}}
\expandafter\def\csname PY@tok@bp\endcsname{\def\PY@tc##1{\textcolor[rgb]{0.00,0.50,0.00}{##1}}}
\expandafter\def\csname PY@tok@fm\endcsname{\def\PY@tc##1{\textcolor[rgb]{0.00,0.00,1.00}{##1}}}
\expandafter\def\csname PY@tok@vc\endcsname{\def\PY@tc##1{\textcolor[rgb]{0.10,0.09,0.49}{##1}}}
\expandafter\def\csname PY@tok@vg\endcsname{\def\PY@tc##1{\textcolor[rgb]{0.10,0.09,0.49}{##1}}}
\expandafter\def\csname PY@tok@vi\endcsname{\def\PY@tc##1{\textcolor[rgb]{0.10,0.09,0.49}{##1}}}
\expandafter\def\csname PY@tok@vm\endcsname{\def\PY@tc##1{\textcolor[rgb]{0.10,0.09,0.49}{##1}}}
\expandafter\def\csname PY@tok@sa\endcsname{\def\PY@tc##1{\textcolor[rgb]{0.73,0.13,0.13}{##1}}}
\expandafter\def\csname PY@tok@sb\endcsname{\def\PY@tc##1{\textcolor[rgb]{0.73,0.13,0.13}{##1}}}
\expandafter\def\csname PY@tok@sc\endcsname{\def\PY@tc##1{\textcolor[rgb]{0.73,0.13,0.13}{##1}}}
\expandafter\def\csname PY@tok@dl\endcsname{\def\PY@tc##1{\textcolor[rgb]{0.73,0.13,0.13}{##1}}}
\expandafter\def\csname PY@tok@s2\endcsname{\def\PY@tc##1{\textcolor[rgb]{0.73,0.13,0.13}{##1}}}
\expandafter\def\csname PY@tok@sh\endcsname{\def\PY@tc##1{\textcolor[rgb]{0.73,0.13,0.13}{##1}}}
\expandafter\def\csname PY@tok@s1\endcsname{\def\PY@tc##1{\textcolor[rgb]{0.73,0.13,0.13}{##1}}}
\expandafter\def\csname PY@tok@mb\endcsname{\def\PY@tc##1{\textcolor[rgb]{0.40,0.40,0.40}{##1}}}
\expandafter\def\csname PY@tok@mf\endcsname{\def\PY@tc##1{\textcolor[rgb]{0.40,0.40,0.40}{##1}}}
\expandafter\def\csname PY@tok@mh\endcsname{\def\PY@tc##1{\textcolor[rgb]{0.40,0.40,0.40}{##1}}}
\expandafter\def\csname PY@tok@mi\endcsname{\def\PY@tc##1{\textcolor[rgb]{0.40,0.40,0.40}{##1}}}
\expandafter\def\csname PY@tok@il\endcsname{\def\PY@tc##1{\textcolor[rgb]{0.40,0.40,0.40}{##1}}}
\expandafter\def\csname PY@tok@mo\endcsname{\def\PY@tc##1{\textcolor[rgb]{0.40,0.40,0.40}{##1}}}
\expandafter\def\csname PY@tok@ch\endcsname{\let\PY@it=\textit\def\PY@tc##1{\textcolor[rgb]{0.25,0.50,0.50}{##1}}}
\expandafter\def\csname PY@tok@cm\endcsname{\let\PY@it=\textit\def\PY@tc##1{\textcolor[rgb]{0.25,0.50,0.50}{##1}}}
\expandafter\def\csname PY@tok@cpf\endcsname{\let\PY@it=\textit\def\PY@tc##1{\textcolor[rgb]{0.25,0.50,0.50}{##1}}}
\expandafter\def\csname PY@tok@c1\endcsname{\let\PY@it=\textit\def\PY@tc##1{\textcolor[rgb]{0.25,0.50,0.50}{##1}}}
\expandafter\def\csname PY@tok@cs\endcsname{\let\PY@it=\textit\def\PY@tc##1{\textcolor[rgb]{0.25,0.50,0.50}{##1}}}

\def\PYZbs{\char`\\}
\def\PYZus{\char`\_}
\def\PYZob{\char`\{}
\def\PYZcb{\char`\}}
\def\PYZca{\char`\^}
\def\PYZam{\char`\&}
\def\PYZlt{\char`\<}
\def\PYZgt{\char`\>}
\def\PYZsh{\char`\#}
\def\PYZpc{\char`\%}
\def\PYZdl{\char`\$}
\def\PYZhy{\char`\-}
\def\PYZsq{\char`\'}
\def\PYZdq{\char`\"}
\def\PYZti{\char`\~}
% for compatibility with earlier versions
\def\PYZat{@}
\def\PYZlb{[}
\def\PYZrb{]}
\makeatother


    % For linebreaks inside Verbatim environment from package fancyvrb. 
    \makeatletter
        \newbox\Wrappedcontinuationbox 
        \newbox\Wrappedvisiblespacebox 
        \newcommand*\Wrappedvisiblespace {\textcolor{red}{\textvisiblespace}} 
        \newcommand*\Wrappedcontinuationsymbol {\textcolor{red}{\llap{\tiny$\m@th\hookrightarrow$}}} 
        \newcommand*\Wrappedcontinuationindent {3ex } 
        \newcommand*\Wrappedafterbreak {\kern\Wrappedcontinuationindent\copy\Wrappedcontinuationbox} 
        % Take advantage of the already applied Pygments mark-up to insert 
        % potential linebreaks for TeX processing. 
        %        {, <, #, %, $, ' and ": go to next line. 
        %        _, }, ^, &, >, - and ~: stay at end of broken line. 
        % Use of \textquotesingle for straight quote. 
        \newcommand*\Wrappedbreaksatspecials {% 
            \def\PYGZus{\discretionary{\char`\_}{\Wrappedafterbreak}{\char`\_}}% 
            \def\PYGZob{\discretionary{}{\Wrappedafterbreak\char`\{}{\char`\{}}% 
            \def\PYGZcb{\discretionary{\char`\}}{\Wrappedafterbreak}{\char`\}}}% 
            \def\PYGZca{\discretionary{\char`\^}{\Wrappedafterbreak}{\char`\^}}% 
            \def\PYGZam{\discretionary{\char`\&}{\Wrappedafterbreak}{\char`\&}}% 
            \def\PYGZlt{\discretionary{}{\Wrappedafterbreak\char`\<}{\char`\<}}% 
            \def\PYGZgt{\discretionary{\char`\>}{\Wrappedafterbreak}{\char`\>}}% 
            \def\PYGZsh{\discretionary{}{\Wrappedafterbreak\char`\#}{\char`\#}}% 
            \def\PYGZpc{\discretionary{}{\Wrappedafterbreak\char`\%}{\char`\%}}% 
            \def\PYGZdl{\discretionary{}{\Wrappedafterbreak\char`\$}{\char`\$}}% 
            \def\PYGZhy{\discretionary{\char`\-}{\Wrappedafterbreak}{\char`\-}}% 
            \def\PYGZsq{\discretionary{}{\Wrappedafterbreak\textquotesingle}{\textquotesingle}}% 
            \def\PYGZdq{\discretionary{}{\Wrappedafterbreak\char`\"}{\char`\"}}% 
            \def\PYGZti{\discretionary{\char`\~}{\Wrappedafterbreak}{\char`\~}}% 
        } 
        % Some characters . , ; ? ! / are not pygmentized. 
        % This macro makes them "active" and they will insert potential linebreaks 
        \newcommand*\Wrappedbreaksatpunct {% 
            \lccode`\~`\.\lowercase{\def~}{\discretionary{\hbox{\char`\.}}{\Wrappedafterbreak}{\hbox{\char`\.}}}% 
            \lccode`\~`\,\lowercase{\def~}{\discretionary{\hbox{\char`\,}}{\Wrappedafterbreak}{\hbox{\char`\,}}}% 
            \lccode`\~`\;\lowercase{\def~}{\discretionary{\hbox{\char`\;}}{\Wrappedafterbreak}{\hbox{\char`\;}}}% 
            \lccode`\~`\:\lowercase{\def~}{\discretionary{\hbox{\char`\:}}{\Wrappedafterbreak}{\hbox{\char`\:}}}% 
            \lccode`\~`\?\lowercase{\def~}{\discretionary{\hbox{\char`\?}}{\Wrappedafterbreak}{\hbox{\char`\?}}}% 
            \lccode`\~`\!\lowercase{\def~}{\discretionary{\hbox{\char`\!}}{\Wrappedafterbreak}{\hbox{\char`\!}}}% 
            \lccode`\~`\/\lowercase{\def~}{\discretionary{\hbox{\char`\/}}{\Wrappedafterbreak}{\hbox{\char`\/}}}% 
            \catcode`\.\active
            \catcode`\,\active 
            \catcode`\;\active
            \catcode`\:\active
            \catcode`\?\active
            \catcode`\!\active
            \catcode`\/\active 
            \lccode`\~`\~ 	
        }
    \makeatother

    \let\OriginalVerbatim=\Verbatim
    \makeatletter
    \renewcommand{\Verbatim}[1][1]{%
        %\parskip\z@skip
        \sbox\Wrappedcontinuationbox {\Wrappedcontinuationsymbol}%
        \sbox\Wrappedvisiblespacebox {\FV@SetupFont\Wrappedvisiblespace}%
        \def\FancyVerbFormatLine ##1{\hsize\linewidth
            \vtop{\raggedright\hyphenpenalty\z@\exhyphenpenalty\z@
                \doublehyphendemerits\z@\finalhyphendemerits\z@
                \strut ##1\strut}%
        }%
        % If the linebreak is at a space, the latter will be displayed as visible
        % space at end of first line, and a continuation symbol starts next line.
        % Stretch/shrink are however usually zero for typewriter font.
        \def\FV@Space {%
            \nobreak\hskip\z@ plus\fontdimen3\font minus\fontdimen4\font
            \discretionary{\copy\Wrappedvisiblespacebox}{\Wrappedafterbreak}
            {\kern\fontdimen2\font}%
        }%
        
        % Allow breaks at special characters using \PYG... macros.
        \Wrappedbreaksatspecials
        % Breaks at punctuation characters . , ; ? ! and / need catcode=\active 	
        \OriginalVerbatim[#1,codes*=\Wrappedbreaksatpunct]%
    }
    \makeatother

    % Exact colors from NB
    \definecolor{incolor}{HTML}{303F9F}
    \definecolor{outcolor}{HTML}{D84315}
    \definecolor{cellborder}{HTML}{CFCFCF}
    \definecolor{cellbackground}{HTML}{F7F7F7}
    
    % prompt
    \makeatletter
    \newcommand{\boxspacing}{\kern\kvtcb@left@rule\kern\kvtcb@boxsep}
    \makeatother
    \newcommand{\prompt}[4]{
        \ttfamily\llap{{\color{#2}[#3]:\hspace{3pt}#4}}\vspace{-\baselineskip}
    }
    

    
    % Prevent overflowing lines due to hard-to-break entities
    \sloppy 
    % Setup hyperref package
    \hypersetup{
      breaklinks=true,  % so long urls are correctly broken across lines
      colorlinks=true,
      urlcolor=urlcolor,
      linkcolor=linkcolor,
      citecolor=citecolor,
      }
    % Slightly bigger margins than the latex defaults
    
    \geometry{verbose,tmargin=1in,bmargin=1in,lmargin=1in,rmargin=1in}
    
    

\begin{document}
    
    \maketitle
    
    

    
    \hypertarget{ux6587ux6625ux30aaux30f3ux30e9ux30a4ux30f3ux306eux8a18ux4e8bux3092ux30b9ux30afux30ecux30a4ux30d4ux30f3ux30b0ux3057ux3066ux30cdux30acux30ddux30b8ux5206ux6790ux3092ux884cux3044ux307eux3059}{%
\subsection{文春オンラインの記事をスクレイピングして、ネガポジ分析を行います}\label{ux6587ux6625ux30aaux30f3ux30e9ux30a4ux30f3ux306eux8a18ux4e8bux3092ux30b9ux30afux30ecux30a4ux30d4ux30f3ux30b0ux3057ux3066ux30cdux30acux30ddux30b8ux5206ux6790ux3092ux884cux3044ux307eux3059}}

\hypertarget{ux30b9ux30afux30ecux30a4ux30d4ux30f3ux30b0ux3068ux306f}{%
\subsubsection{スクレイピングとは}\label{ux30b9ux30afux30ecux30a4ux30d4ux30f3ux30b0ux3068ux306f}}

スクレイピングを利用することでネット上の様々な情報を取得することができます。\\
今回はPythonのコードを利用して記事を取得しています。\\
\texttt{BeautifulSoup}などを利用して、HTMLやCSSの情報を指定して情報を抽出します。

\hypertarget{ux30cdux30acux30ddux30b8ux5206ux6790ux306bux3064ux3044ux3066}{%
\subsubsection{ネガポジ分析について}\label{ux30cdux30acux30ddux30b8ux5206ux6790ux306bux3064ux3044ux3066}}

記事の内容がネガティブなのかポジティブなのかを\href{http://www.lr.pi.titech.ac.jp/~takamura/pndic_ja.html}{単語感情極性対応表}を基準にして数値化してみようと思います。\\
単語感情極性対応表は日本語の単語に対してネガポジ度を-1〜1で定義してあります。

喜び:よろこび:名詞:0.998861\\
厳しい:きびしい:形容詞:-0.999755

などです。

\hypertarget{ux53c2ux8003ux6587ux732e}{%
\subsubsection{参考文献}\label{ux53c2ux8003ux6587ux732e}}

スクレイピングは\href{https://www.udemy.com/course/python-web-scraping-with-beautifulsoup-selenium-requests/}{udemy講座}を利用しました。\\
ネガポジ分析は\href{https://qiita.com/g-k/items/e49f68d7e2fed6e300ea}{こちらの記事}を参考にさせていただきました。\\
また、今回分析対象としたのは\href{https://bunshun.jp/}{文春オンライン}です。

    では作業に取り掛かりましょう。\\
まずは必要なライブラリをインポートします。

    \begin{tcolorbox}[breakable, size=fbox, boxrule=1pt, pad at break*=1mm,colback=cellbackground, colframe=cellborder]
\prompt{In}{incolor}{1}{\boxspacing}
\begin{Verbatim}[commandchars=\\\{\}]
\PY{k+kn}{import} \PY{n+nn}{requests}
\PY{k+kn}{from} \PY{n+nn}{bs4} \PY{k+kn}{import} \PY{n}{BeautifulSoup}
\PY{k+kn}{import} \PY{n+nn}{re}
\PY{k+kn}{import} \PY{n+nn}{itertools}
\PY{k+kn}{import} \PY{n+nn}{pandas} \PY{k}{as} \PY{n+nn}{pd}\PY{o}{,} \PY{n+nn}{numpy} \PY{k}{as} \PY{n+nn}{np}
\PY{k+kn}{import} \PY{n+nn}{os}
\PY{k+kn}{import} \PY{n+nn}{glob}
\PY{k+kn}{import} \PY{n+nn}{pathlib}
\PY{k+kn}{import} \PY{n+nn}{re}
\PY{k+kn}{import} \PY{n+nn}{janome}
\PY{k+kn}{import} \PY{n+nn}{jaconv}
\PY{k+kn}{from} \PY{n+nn}{janome}\PY{n+nn}{.}\PY{n+nn}{tokenizer} \PY{k+kn}{import} \PY{n}{Tokenizer}
\PY{k+kn}{from} \PY{n+nn}{janome}\PY{n+nn}{.}\PY{n+nn}{analyzer} \PY{k+kn}{import} \PY{n}{Analyzer}
\PY{k+kn}{from} \PY{n+nn}{janome}\PY{n+nn}{.}\PY{n+nn}{charfilter} \PY{k+kn}{import} \PY{o}{*}
\end{Verbatim}
\end{tcolorbox}

    \hypertarget{ux307eux305aux306fux30b9ux30afux30ecux30a4ux30d4ux30f3ux30b0ux306eux6e96ux5099ux3067ux3059}{%
\subsubsection{まずはスクレイピングの準備です。}\label{ux307eux305aux306fux30b9ux30afux30ecux30a4ux30d4ux30f3ux30b0ux306eux6e96ux5099ux3067ux3059}}

URLを取得して\texttt{requests}と\texttt{BeautifulSoup}を適用します。

    \begin{tcolorbox}[breakable, size=fbox, boxrule=1pt, pad at break*=1mm,colback=cellbackground, colframe=cellborder]
\prompt{In}{incolor}{2}{\boxspacing}
\begin{Verbatim}[commandchars=\\\{\}]
\PY{n}{url} \PY{o}{=} \PY{l+s+s2}{\PYZdq{}}\PY{l+s+s2}{https://bunshun.jp/}\PY{l+s+s2}{\PYZdq{}} \PY{c+c1}{\PYZsh{} urlに文春オンラインのリンクを格納}
\PY{n}{res} \PY{o}{=} \PY{n}{requests}\PY{o}{.}\PY{n}{get}\PY{p}{(}\PY{n}{url}\PY{p}{)} \PY{c+c1}{\PYZsh{} requests.get()を用いてurlをresに格納}
\PY{n}{soup} \PY{o}{=} \PY{n}{BeautifulSoup}\PY{p}{(}\PY{n}{res}\PY{o}{.}\PY{n}{text}\PY{p}{,} \PY{l+s+s2}{\PYZdq{}}\PY{l+s+s2}{html.parser}\PY{l+s+s2}{\PYZdq{}}\PY{p}{)} \PY{c+c1}{\PYZsh{} ここでBeautifulSoupを用いてスクレイピングの準備ができました。}
\end{Verbatim}
\end{tcolorbox}

    \hypertarget{ux8a18ux4e8bux4e00ux89a7ux3092ux53d6ux5f97ux3057ux3066ux30bfux30a4ux30c8ux30ebux3068urlux3092ux53d6ux5f97}{%
\subsubsection{記事一覧を取得して、タイトルとURLを取得}\label{ux8a18ux4e8bux4e00ux89a7ux3092ux53d6ux5f97ux3057ux3066ux30bfux30a4ux30c8ux30ebux3068urlux3092ux53d6ux5f97}}

基本的には記事が\texttt{li}要素として並んでおり、その親要素は\texttt{ul}であるというパターンが多いです。\\
for文を使って記事一覧からタイトルとURLを取得しています。

    \begin{tcolorbox}[breakable, size=fbox, boxrule=1pt, pad at break*=1mm,colback=cellbackground, colframe=cellborder]
\prompt{In}{incolor}{3}{\boxspacing}
\begin{Verbatim}[commandchars=\\\{\}]
\PY{n}{elems} \PY{o}{=} \PY{n}{soup}\PY{o}{.}\PY{n}{select}\PY{p}{(}\PY{l+s+s2}{\PYZdq{}}\PY{l+s+s2}{ul}\PY{l+s+s2}{\PYZdq{}}\PY{p}{)} \PY{c+c1}{\PYZsh{} 記事のリストがli要素として並んでいたので、その親要素であるulを指定しています。}
\PY{n}{title\PYZus{}list} \PY{o}{=} \PY{p}{[}\PY{p}{]} \PY{c+c1}{\PYZsh{} 記事のタイトルを格納するリスト}
\PY{n}{url\PYZus{}list} \PY{o}{=} \PY{p}{[}\PY{p}{]} \PY{c+c1}{\PYZsh{} 記事のURLを格納するリスト}
\PY{k}{for} \PY{n}{sibling} \PY{o+ow}{in} \PY{n}{elems}\PY{p}{[}\PY{l+m+mi}{3}\PY{p}{]}\PY{p}{:} \PY{c+c1}{\PYZsh{} elems[3]に欲しいリストがありました。このfor分により記事のリストから記事のタイトルとURLを取得し、それぞれリストに格納します。}
    \PY{k}{if} \PY{n}{sibling} \PY{o}{!=} \PY{l+s+s2}{\PYZdq{}}\PY{l+s+se}{\PYZbs{}n}\PY{l+s+s2}{\PYZdq{}}\PY{p}{:} \PY{c+c1}{\PYZsh{} 改行が含まれていたので除外}
        \PY{n+nb}{print}\PY{p}{(}\PY{n}{sibling}\PY{o}{.}\PY{n}{h3}\PY{o}{.}\PY{n}{string}\PY{p}{)} \PY{c+c1}{\PYZsh{} タイトルはh3タグに入っていました。}
        \PY{n}{title\PYZus{}list}\PY{o}{.}\PY{n}{append}\PY{p}{(}\PY{n}{sibling}\PY{o}{.}\PY{n}{h3}\PY{o}{.}\PY{n}{string}\PY{o}{.}\PY{n}{replace}\PY{p}{(}\PY{l+s+s1}{\PYZsq{}}\PY{l+s+se}{\PYZbs{}u3000}\PY{l+s+s1}{\PYZsq{}}\PY{p}{,} \PY{l+s+s1}{\PYZsq{}}\PY{l+s+s1}{ }\PY{l+s+s1}{\PYZsq{}}\PY{p}{)}\PY{p}{)} \PY{c+c1}{\PYZsh{} \PYZbs{}u3000が入っている部分があったので空白に変換}
        \PY{n+nb}{print}\PY{p}{(}\PY{n}{url} \PY{o}{+} \PY{n}{sibling}\PY{o}{.}\PY{n}{h3}\PY{o}{.}\PY{n}{a}\PY{p}{[}\PY{l+s+s2}{\PYZdq{}}\PY{l+s+s2}{href}\PY{l+s+s2}{\PYZdq{}}\PY{p}{]}\PY{p}{)} \PY{c+c1}{\PYZsh{} aタグのhref属性にリンクが格納されていました。}
        \PY{n}{url\PYZus{}list}\PY{o}{.}\PY{n}{append}\PY{p}{(}\PY{n}{url} \PY{o}{+} \PY{n}{sibling}\PY{o}{.}\PY{n}{h3}\PY{o}{.}\PY{n}{a}\PY{p}{[}\PY{l+s+s2}{\PYZdq{}}\PY{l+s+s2}{href}\PY{l+s+s2}{\PYZdq{}}\PY{p}{]}\PY{p}{)} \PY{c+c1}{\PYZsh{} 上記で取得したurl以下の部分が格納されていたので足しています。}
\end{Verbatim}
\end{tcolorbox}

    \begin{Verbatim}[commandchars=\\\{\}]
ジャニーズ元MADEリーダー・稲葉光(29)が元Berryz工房アイドルと“渋谷ホテルデート”《スクープ撮》
https://bunshun.jp//articles/-/40869
2050年「CO2排出量ゼロ」を宣言したJR東日本の戦略とは?
https://bunshun.jp//articles/-/40083
「365日24時間働こう」……ワタミの“思想教育”はいまも続いていた
https://bunshun.jp//articles/-/40843
“縦割り行政”の弊害だ! 橋下徹が語る「Go Toキャンペーンは何が間違っているのか」
https://bunshun.jp//articles/-/40877
警察官に唾吐き公務執行妨害で逮捕 読売新聞がソウル支局記者を懲戒処分
https://bunshun.jp//articles/-/40868
同性愛差別の足立区議宛てに「とんでもない思い違いです」…81歳祖母の手紙が教えてくれたこと
https://bunshun.jp//articles/-/40826
家系ラーメン“のれん分け戦争”「吉村家vs.六角家」裏切りと屈服の黒歴史〈六角家破産〉
https://bunshun.jp//articles/-/40752
《平手派また卒業》欅坂46・佐藤詩織が書いた「活動のなかで悲しかったこと」 櫻坂46の“急勾配”を運営も懸念
https://bunshun.jp//articles/-/40862
安倍政権最大の功績は“アイヌ博物館”だった? 200億円をブチ込んだ「ウポポイ」の虚実
https://bunshun.jp//articles/-/40841
「長さ」ではなく…美容師が教える、髪を切るときに言うといい「意外な言葉」
https://bunshun.jp//articles/-/40694
    \end{Verbatim}

    \hypertarget{ux8a18ux4e8bux306f1ux30daux30fcux30b8ux3067ux306fux7d42ux308fux3089ux306aux3044ux5834ux5408ux304cux3042ux308aux307eux30591ux30daux30fcux30b8ux305aux3064ux9077ux79fbux3057ux3066ux30eaux30f3ux30afux3092ux53d6ux5f97ux3057ux307eux3059}{%
\subsubsection{記事は1ページでは終わらない場合があります。1ページずつ遷移してリンクを取得します。}\label{ux8a18ux4e8bux306f1ux30daux30fcux30b8ux3067ux306fux7d42ux308fux3089ux306aux3044ux5834ux5408ux304cux3042ux308aux307eux30591ux30daux30fcux30b8ux305aux3064ux9077ux79fbux3057ux3066ux30eaux30f3ux30afux3092ux53d6ux5f97ux3057ux307eux3059}}

リンク一覧を作成するために、while文を作って次のページが表示されていればそのURLを取得してページに遷移し、\\
遷移したページにも次のページへのリンクがあれば取得して遷移するというループを回します。\\
こうすることにより、1タイトルのニュースに関する全てのページに関するリンク一覧を作成できます。

    \begin{tcolorbox}[breakable, size=fbox, boxrule=1pt, pad at break*=1mm,colback=cellbackground, colframe=cellborder]
\prompt{In}{incolor}{4}{\boxspacing}
\begin{Verbatim}[commandchars=\\\{\}]
\PY{n}{news\PYZus{}list} \PY{o}{=} \PY{p}{[}\PY{p}{]} \PY{c+c1}{\PYZsh{} 全てのニュース記事のリンクをここに格納します。}
\PY{k}{for} \PY{n}{pickup\PYZus{}link} \PY{o+ow}{in} \PY{n}{url\PYZus{}list}\PY{p}{:} \PY{c+c1}{\PYZsh{} このfor文でURLリストからURLを取り出します。}
    \PY{n}{news} \PY{o}{=} \PY{p}{[}\PY{p}{]} \PY{c+c1}{\PYZsh{} ニュース記事はページごとに分かれているため、このリストに各ページのリンクを格納します。}
    \PY{n}{news}\PY{o}{.}\PY{n}{append}\PY{p}{(}\PY{n}{pickup\PYZus{}link}\PY{p}{)} \PY{c+c1}{\PYZsh{} 最初のリンクを格納}
    \PY{n}{pickup\PYZus{}res} \PY{o}{=} \PY{n}{requests}\PY{o}{.}\PY{n}{get}\PY{p}{(}\PY{n}{pickup\PYZus{}link}\PY{p}{)} \PY{c+c1}{\PYZsh{} requests.get()を用いてリンクからページを取得}
    \PY{n}{pickup\PYZus{}soup} \PY{o}{=} \PY{n}{BeautifulSoup}\PY{p}{(}\PY{n}{pickup\PYZus{}res}\PY{o}{.}\PY{n}{text}\PY{p}{,} \PY{l+s+s2}{\PYZdq{}}\PY{l+s+s2}{html.parser}\PY{l+s+s2}{\PYZdq{}}\PY{p}{)} \PY{c+c1}{\PYZsh{} BeautifulSoupを適用}
    \PY{k}{while} \PY{k+kc}{True}\PY{p}{:} \PY{c+c1}{\PYZsh{} このwhile文では次のページへのリンクがあればそのリンクを取得し、そのページへ遷移するというループを回します。}
        \PY{k}{try}\PY{p}{:} \PY{c+c1}{\PYZsh{} 遷移した先で次のページへのリンクがあれば永遠にこのループを繰り返します。}
            \PY{n}{next\PYZus{}link} \PY{o}{=} \PY{n}{pickup\PYZus{}soup}\PY{o}{.}\PY{n}{find}\PY{p}{(}\PY{l+s+s2}{\PYZdq{}}\PY{l+s+s2}{a}\PY{l+s+s2}{\PYZdq{}}\PY{p}{,} \PY{n}{class\PYZus{}}\PY{o}{=}\PY{l+s+s2}{\PYZdq{}}\PY{l+s+s2}{next menu\PYZhy{}link ga\PYZus{}tracking}\PY{l+s+s2}{\PYZdq{}}\PY{p}{)}\PY{p}{[}\PY{l+s+s2}{\PYZdq{}}\PY{l+s+s2}{href}\PY{l+s+s2}{\PYZdq{}}\PY{p}{]} \PY{c+c1}{\PYZsh{} next menu\PYZhy{}link ga\PYZus{}trackingというクラスを持つaタグのhref属性が次のページへのリンクでした。}
            \PY{n}{next\PYZus{}link} \PY{o}{=} \PY{n}{url} \PY{o}{+} \PY{n}{next\PYZus{}link}
            \PY{n}{next\PYZus{}res} \PY{o}{=} \PY{n}{requests}\PY{o}{.}\PY{n}{get}\PY{p}{(}\PY{n}{next\PYZus{}link}\PY{p}{)} \PY{c+c1}{\PYZsh{} requests.get()とBeautifulSoupを用いて遷移先のページ情報を取得します。}
            \PY{n}{pickup\PYZus{}soup} \PY{o}{=} \PY{n}{BeautifulSoup}\PY{p}{(}\PY{n}{next\PYZus{}res}\PY{o}{.}\PY{n}{text}\PY{p}{,} \PY{l+s+s2}{\PYZdq{}}\PY{l+s+s2}{html.parser}\PY{l+s+s2}{\PYZdq{}}\PY{p}{)}
            \PY{n}{news}\PY{o}{.}\PY{n}{append}\PY{p}{(}\PY{n}{next\PYZus{}link}\PY{p}{)} \PY{c+c1}{\PYZsh{} newsに各ページ情報を追加します。}
        \PY{k}{except} \PY{n+ne}{Exception}\PY{p}{:} \PY{c+c1}{\PYZsh{} 次のページへのリンクが無ければここの処理が行われます。}
            \PY{n}{news\PYZus{}list}\PY{o}{.}\PY{n}{append}\PY{p}{(}\PY{n}{news}\PY{p}{)} \PY{c+c1}{\PYZsh{} タイトル内の全ての記事をURLがnewsに格納されたのでそれをnews\PYZus{}listに格納します。}
            \PY{k}{break}
\PY{n}{display}\PY{p}{(}\PY{n}{news\PYZus{}list}\PY{p}{)} \PY{c+c1}{\PYZsh{} 作成したURLリストを表示します。}
\end{Verbatim}
\end{tcolorbox}

    
    \begin{verbatim}
[['https://bunshun.jp//articles/-/40869',
  'https://bunshun.jp//articles/-/40869?page=2',
  'https://bunshun.jp//articles/-/40869?page=3',
  'https://bunshun.jp//articles/-/40869?page=4'],
 ['https://bunshun.jp//articles/-/40083'],
 ['https://bunshun.jp//articles/-/40843',
  'https://bunshun.jp//articles/-/40843?page=2',
  'https://bunshun.jp//articles/-/40843?page=3',
  'https://bunshun.jp//articles/-/40843?page=4'],
 ['https://bunshun.jp//articles/-/40877',
  'https://bunshun.jp//articles/-/40877?page=2'],
 ['https://bunshun.jp//articles/-/40868',
  'https://bunshun.jp//articles/-/40868?page=2'],
 ['https://bunshun.jp//articles/-/40826',
  'https://bunshun.jp//articles/-/40826?page=2',
  'https://bunshun.jp//articles/-/40826?page=3',
  'https://bunshun.jp//articles/-/40826?page=4'],
 ['https://bunshun.jp//articles/-/40752',
  'https://bunshun.jp//articles/-/40752?page=2',
  'https://bunshun.jp//articles/-/40752?page=3',
  'https://bunshun.jp//articles/-/40752?page=4'],
 ['https://bunshun.jp//articles/-/40862',
  'https://bunshun.jp//articles/-/40862?page=2',
  'https://bunshun.jp//articles/-/40862?page=3'],
 ['https://bunshun.jp//articles/-/40841',
  'https://bunshun.jp//articles/-/40841?page=2',
  'https://bunshun.jp//articles/-/40841?page=3',
  'https://bunshun.jp//articles/-/40841?page=4',
  'https://bunshun.jp//articles/-/40841?page=5'],
 ['https://bunshun.jp//articles/-/40694',
  'https://bunshun.jp//articles/-/40694?page=2',
  'https://bunshun.jp//articles/-/40694?page=3',
  'https://bunshun.jp//articles/-/40694?page=4']]
    \end{verbatim}

    
    \hypertarget{ux5148ux307bux3069ux306eux30b3ux30fcux30c9ux3067urlux30eaux30b9ux30c8ux3092ux4f5cux6210ux3067ux304dux305fux306eux3067ux305dux306eux30eaux30f3ux30afux3092ux8fbfux3063ux3066ux8a18ux4e8bux672cux6587ux3092ux53d6ux5f97ux3057ux3066ux3044ux304dux307eux3059}{%
\subsubsection{先ほどのコードでURLリストを作成できたので、そのリンクを辿って記事本文を取得していきます。}\label{ux5148ux307bux3069ux306eux30b3ux30fcux30c9ux3067urlux30eaux30b9ux30c8ux3092ux4f5cux6210ux3067ux304dux305fux306eux3067ux305dux306eux30eaux30f3ux30afux3092ux8fbfux3063ux3066ux8a18ux4e8bux672cux6587ux3092ux53d6ux5f97ux3057ux3066ux3044ux304dux307eux3059}}

\texttt{.text}を適用することで本文のみを取得できるのですが、細かくfor文を回して\texttt{.text}を適用しています。\\
そのためいくつかの空文字(or空リスト)を作成して格納しながら本文を格納したリストを作成していきました。

    \begin{tcolorbox}[breakable, size=fbox, boxrule=1pt, pad at break*=1mm,colback=cellbackground, colframe=cellborder]
\prompt{In}{incolor}{5}{\boxspacing}
\begin{Verbatim}[commandchars=\\\{\}]
\PY{n}{news\PYZus{}page\PYZus{}list} \PY{o}{=} \PY{p}{[}\PY{p}{]} \PY{c+c1}{\PYZsh{} ここに全ての記事の本文を格納します。}
\PY{k}{for} \PY{n}{news\PYZus{}links} \PY{o+ow}{in} \PY{n}{news\PYZus{}list}\PY{p}{:} \PY{c+c1}{\PYZsh{} このfor文でURLのリストからあるタイトルのリンクリストを取り出します。}
    \PY{n}{news\PYZus{}page} \PY{o}{=} \PY{l+s+s1}{\PYZsq{}}\PY{l+s+s1}{\PYZsq{}} \PY{c+c1}{\PYZsh{} ここに各ページから取得した本文を追加していきます。}
    \PY{k}{for} \PY{n}{news\PYZus{}link} \PY{o+ow}{in} \PY{n}{news\PYZus{}links}\PY{p}{:} \PY{c+c1}{\PYZsh{} タイトルのリンクリストからリンクを一つずつ取り出します。}
        \PY{n}{news\PYZus{}res} \PY{o}{=} \PY{n}{requests}\PY{o}{.}\PY{n}{get}\PY{p}{(}\PY{n}{news\PYZus{}link}\PY{p}{)} \PY{c+c1}{\PYZsh{} requests.get()とBeautifulSoupを利用して記事の情報を取得します。}
        \PY{n}{news\PYZus{}soup} \PY{o}{=} \PY{n}{BeautifulSoup}\PY{p}{(}\PY{n}{news\PYZus{}res}\PY{o}{.}\PY{n}{text}\PY{p}{,} \PY{l+s+s2}{\PYZdq{}}\PY{l+s+s2}{html.parser}\PY{l+s+s2}{\PYZdq{}}\PY{p}{)} 
        \PY{n}{news\PYZus{}soup} \PY{o}{=} \PY{n}{news\PYZus{}soup}\PY{o}{.}\PY{n}{find}\PY{p}{(}\PY{n}{class\PYZus{}}\PY{o}{=}\PY{n}{re}\PY{o}{.}\PY{n}{compile}\PY{p}{(}\PY{l+s+s2}{\PYZdq{}}\PY{l+s+s2}{article\PYZhy{}body}\PY{l+s+s2}{\PYZdq{}}\PY{p}{)}\PY{p}{)}\PY{o}{.}\PY{n}{find\PYZus{}all}\PY{p}{(}\PY{l+s+s2}{\PYZdq{}}\PY{l+s+s2}{p}\PY{l+s+s2}{\PYZdq{}}\PY{p}{)} \PY{c+c1}{\PYZsh{} article\PYZhy{}bodyというidを持つタグの直下のpタグに本文が格納されていました。}
        \PY{n}{news\PYZus{}phrase} \PY{o}{=} \PY{l+s+s1}{\PYZsq{}}\PY{l+s+s1}{\PYZsq{}} \PY{c+c1}{\PYZsh{} そのページの本文のフレーズを格納}
        \PY{k}{for} \PY{n}{news} \PY{o+ow}{in} \PY{n}{news\PYZus{}soup}\PY{p}{:} \PY{c+c1}{\PYZsh{} for文で回すことでtextを適用して本文フレーズのみを取得できました。}
            \PY{n}{news\PYZus{}phrase} \PY{o}{+}\PY{o}{=} \PY{n}{news}\PY{o}{.}\PY{n}{text}\PY{o}{.}\PY{n}{replace}\PY{p}{(}\PY{l+s+s1}{\PYZsq{}}\PY{l+s+se}{\PYZbs{}u3000}\PY{l+s+s1}{\PYZsq{}}\PY{p}{,} \PY{l+s+s1}{\PYZsq{}}\PY{l+s+s1}{ }\PY{l+s+s1}{\PYZsq{}}\PY{p}{)} \PY{c+c1}{\PYZsh{} 取得したフレーズを追加。文字列なので+で追加できました。}
        \PY{n}{news\PYZus{}page} \PY{o}{+}\PY{o}{=} \PY{n}{news\PYZus{}phrase} \PY{c+c1}{\PYZsh{} 1ページ分のフレーズが取得できらたらnew\PYZus{}pageに追加}
    \PY{n}{news\PYZus{}page\PYZus{}list}\PY{o}{.}\PY{n}{append}\PY{p}{(}\PY{n}{news\PYZus{}page}\PY{p}{)} \PY{c+c1}{\PYZsh{} 一つのタイトルに対する全ての本文がnew\PYZus{}pageに格納されたらnews\PYZus{}page\PYZus{}listに追加。これはリスト型なのでappendを使用。}
\end{Verbatim}
\end{tcolorbox}

    \begin{tcolorbox}[breakable, size=fbox, boxrule=1pt, pad at break*=1mm,colback=cellbackground, colframe=cellborder]
\prompt{In}{incolor}{6}{\boxspacing}
\begin{Verbatim}[commandchars=\\\{\}]
\PY{k}{for} \PY{n}{i} \PY{o+ow}{in} \PY{n+nb}{range}\PY{p}{(}\PY{l+m+mi}{3}\PY{p}{)}\PY{p}{:} \PY{c+c1}{\PYZsh{} 取得した本文の一部を表示してみます。うまく取得できたようです。}
    \PY{n+nb}{print}\PY{p}{(}\PY{n}{news\PYZus{}page\PYZus{}list}\PY{p}{[}\PY{n}{i}\PY{p}{]}\PY{p}{[}\PY{p}{:}\PY{l+m+mi}{500}\PY{p}{]}\PY{p}{,} \PY{n}{end}\PY{o}{=}\PY{l+s+s2}{\PYZdq{}}\PY{l+s+se}{\PYZbs{}n}\PY{l+s+se}{\PYZbs{}n}\PY{l+s+s2}{\PYZdq{}}\PY{p}{)}
\end{Verbatim}
\end{tcolorbox}

    \begin{Verbatim}[commandchars=\\\{\}]
 ジャニーズJr.内の人気ユニット「宇宙Six」の山本亮太(30)が違法な闇スロット店に通っていたという文春オンラインの報道を受け、10月2日、ジャニーズ事務所
は山本との専属契約を解除。その2日後、「宇宙Six」は解散と発表された。「山本が抜けた『宇宙Six』には、江田剛(33)、松本幸大(31)、原嘉孝(25)の3人
が残っていました。一部報道では、メンバーから解散の申し入れがあったかのように書かれていますが、彼らの本音は『3人でもグループの活動を続けていきたかった』。でも、
事務所のある幹部が『甘いよ』『わかってるよね』と……。解散は仕方ないにしても、せめて、ファンのために予定していたコンサートだけでもやりたいと3人は懇願したのです
が、それも結局許されなかった。それだけ“あの方”は本気でした」(ジャニーズ関係者)“あの方”とは、滝沢秀明副社長(38)のこと。昨年9月にジャニーズ事務所副社長
に就任した滝沢氏はジャニーズJr.たちの育成も担当している。これまで、滝沢副社長は所属タレントにスキャンダルが報じられると、それが違法行為だった場合はもちろんの
こと、女性スキャンダルに対しても厳正な処

地球温暖化に歯止めをかけるため、一日も早い脱炭素社会の実現を。JR東日本では今年5月、2050年度のCO2
排出量“実質ゼロ”をめざす「ゼロカーボン・チャレンジ2050」をスタートさせた。
※「実質ゼロ」…排出されるCO2と同じ量のCO2を最先端技術等により吸収・回収・利用して事実上ゼロにすること JR東日本は2018年7月に発表したグループ経営ビ
ジョン『変革2027』で、社会の一員として果たすべき責任を全うするため、持続可能な開発目標SDGsに積極的にコミットすることを表明。さらに今回、環境長期目標「ゼ
ロカーボン・チャレンジ2050」をスタートさせた。2050年度までにCO2排出量「実質ゼロ」を実現するための方策とはどのようなものなのか、プロジェクトの責任者で
あるJR東日本 総合企画本部
経営企画部次長の笠井浩司さんに聞いた。――今回新たに「ゼロカーボン・チャレンジ2050」をスタートさせた背景をお聞かせください。笠井 現在、地球温暖化防止に関し
て、平均気温上昇を産業革命以前の+1.5℃に抑えるというのが世界の潮流になっています。当社では『変革2027』の中で「2030年度にCO2を

 ワタミ株式会社の労働問題に関する告発が続いている。10月2日、「ワタミの宅食」営業所の所長が、労働基準監督署からの残業代未払いの是正勧告、月175時間を超える
長時間労働、上司によるタイムカードの改ざんを次々と公表したのだ。「ホワイト企業」宣伝のワタミで月175時間の残業
残業代未払いで労基署から是正勧告ワタミがホワイト企業になれなかった理由は? 勝手に勤怠「改ざん」システムも Aさんは長時間労働の末、昼夜の感覚がなくなり、「この
まま寝たら、もう目が覚めないのではないか」と恐怖を抱きながら生活するほどだった。「あのまま働いていたら、死んでいた」とAさんは断言する。現在は、精神疾患を発症し
、労災申請をしながら休業中だ。 しかし、Aさんは命の危険を感じていながら、なぜ過酷な仕事を続けてしまったのだろうか。その背景には、労働者の意識に働きかけ、過酷な
労働を受け入れさせてしまう、ワタミによる「思想教育」のシステムがあった。「こんなにいい仕事をしているんだから、苦しくても頑張ろう」「苦しいことも、苦しくない。む
しろ自分の力になる」 Aさんは過重労働の最中、そう自分に言い聞かせていた。 実際、Aさ

    \end{Verbatim}

    \hypertarget{ux30b9ux30afux30ecux30a4ux30d4ux30f3ux30b0ux306bux3088ux3063ux3066ux5f97ux305fux4ecaux307eux3067ux306eux60c5ux5831ux3092ux4e00ux3064ux306edataframeux306bux683cux7d0dux3057ux307eux3059}{%
\subsubsection{スクレイピングによって得た今までの情報を一つのDataFrameに格納します。}\label{ux30b9ux30afux30ecux30a4ux30d4ux30f3ux30b0ux306bux3088ux3063ux3066ux5f97ux305fux4ecaux307eux3067ux306eux60c5ux5831ux3092ux4e00ux3064ux306edataframeux306bux683cux7d0dux3057ux307eux3059}}

ここまでできればあとはデータを加工してネガポジ分析をするだけです!

    \begin{tcolorbox}[breakable, size=fbox, boxrule=1pt, pad at break*=1mm,colback=cellbackground, colframe=cellborder]
\prompt{In}{incolor}{7}{\boxspacing}
\begin{Verbatim}[commandchars=\\\{\}]
\PY{n}{new\PYZus{}no\PYZus{}list} \PY{o}{=} \PY{p}{[}\PY{n}{x} \PY{k}{for} \PY{n}{x} \PY{o+ow}{in} \PY{n+nb}{range}\PY{p}{(}\PY{n+nb}{len}\PY{p}{(}\PY{n}{title\PYZus{}list}\PY{p}{)}\PY{p}{)}\PY{p}{]} \PY{c+c1}{\PYZsh{} あとで使うのでニュースNo.を作成}
\PY{n}{newslist} \PY{o}{=} \PY{n}{np}\PY{o}{.}\PY{n}{array}\PY{p}{(}\PY{p}{[}\PY{n}{new\PYZus{}no\PYZus{}list}\PY{p}{,} \PY{n}{title\PYZus{}list}\PY{p}{,} \PY{n}{url\PYZus{}list}\PY{p}{,} \PY{n}{news\PYZus{}page\PYZus{}list}\PY{p}{]}\PY{p}{)}\PY{o}{.}\PY{n}{T} \PY{c+c1}{\PYZsh{} DataFrameに格納する準備として、np.arrayのリストに格納して転置しておく。}
\PY{n}{newslist} \PY{o}{=} \PY{n}{pd}\PY{o}{.}\PY{n}{DataFrame}\PY{p}{(}\PY{n}{newslist}\PY{p}{,} \PY{n}{columns}\PY{o}{=}\PY{p}{[}\PY{l+s+s1}{\PYZsq{}}\PY{l+s+s1}{ニュースNo.}\PY{l+s+s1}{\PYZsq{}}\PY{p}{,} \PY{l+s+s1}{\PYZsq{}}\PY{l+s+s1}{title}\PY{l+s+s1}{\PYZsq{}}\PY{p}{,} \PY{l+s+s1}{\PYZsq{}}\PY{l+s+s1}{url}\PY{l+s+s1}{\PYZsq{}}\PY{p}{,} \PY{l+s+s1}{\PYZsq{}}\PY{l+s+s1}{news\PYZus{}page\PYZus{}list}\PY{l+s+s1}{\PYZsq{}}\PY{p}{]}\PY{p}{)} \PY{c+c1}{\PYZsh{} カラム名を指定してDataFrameに格納}
\PY{n}{newslist} \PY{o}{=} \PY{n}{newslist}\PY{o}{.}\PY{n}{astype}\PY{p}{(}\PY{p}{\PYZob{}}\PY{l+s+s1}{\PYZsq{}}\PY{l+s+s1}{ニュースNo.}\PY{l+s+s1}{\PYZsq{}}\PY{p}{:}\PY{l+s+s1}{\PYZsq{}}\PY{l+s+s1}{int64}\PY{l+s+s1}{\PYZsq{}}\PY{p}{\PYZcb{}}\PY{p}{)} \PY{c+c1}{\PYZsh{} あとでテーブルを結合するためにニュースNo.をint64型に変換}
\end{Verbatim}
\end{tcolorbox}

    \begin{tcolorbox}[breakable, size=fbox, boxrule=1pt, pad at break*=1mm,colback=cellbackground, colframe=cellborder]
\prompt{In}{incolor}{8}{\boxspacing}
\begin{Verbatim}[commandchars=\\\{\}]
\PY{n}{newslist}
\end{Verbatim}
\end{tcolorbox}

            \begin{tcolorbox}[breakable, size=fbox, boxrule=.5pt, pad at break*=1mm, opacityfill=0]
\prompt{Out}{outcolor}{8}{\boxspacing}
\begin{Verbatim}[commandchars=\\\{\}]
   ニュースNo.                                              title  \textbackslash{}
0        0  ジャニーズ元MADEリーダー・稲葉光(29)が元Berryz工房アイドルと“渋谷ホテルデート{\ldots}
1        1                    2050年「CO2排出量ゼロ」を宣言したJR東日本の戦略とは?
2        2                 「365日24時間働こう」……ワタミの“思想教育”はいまも続いていた
3        3        “縦割り行政”の弊害だ! 橋下徹が語る「Go Toキャンペーンは何が間違っているのか」
4        4                 警察官に唾吐き公務執行妨害で逮捕 読売新聞がソウル支局記者を懲戒処分
5        5      同性愛差別の足立区議宛てに「とんでもない思い違いです」…81歳祖母の手紙が教えてくれたこと
6        6        家系ラーメン“のれん分け戦争”「吉村家vs.六角家」裏切りと屈服の黒歴史〈六角家破産〉
7        7  《平手派また卒業》欅坂46・佐藤詩織が書いた「活動のなかで悲しかったこと」 櫻坂46の“急勾{\ldots}
8        8        安倍政権最大の功績は“アイヌ博物館”だった? 200億円をブチ込んだ「ウポポイ」の虚実
9        9               「長さ」ではなく…美容師が教える、髪を切るときに言うといい「意外な言葉」

                                    url  \textbackslash{}
0  https://bunshun.jp//articles/-/40869
1  https://bunshun.jp//articles/-/40083
2  https://bunshun.jp//articles/-/40843
3  https://bunshun.jp//articles/-/40877
4  https://bunshun.jp//articles/-/40868
5  https://bunshun.jp//articles/-/40826
6  https://bunshun.jp//articles/-/40752
7  https://bunshun.jp//articles/-/40862
8  https://bunshun.jp//articles/-/40841
9  https://bunshun.jp//articles/-/40694

                                      news\_page\_list
0   ジャニーズJr.内の人気ユニット「宇宙Six」の山本亮太(30)が違法な闇スロット店に通っ{\ldots}
1  地球温暖化に歯止めをかけるため、一日も早い脱炭素社会の実現を。JR東日本では今年5月、205{\ldots}
2   ワタミ株式会社の労働問題に関する告発が続いている。10月2日、「ワタミの宅食」営業所の所長{\ldots}
3  「規制改革」「行政改革」「縦割り打破」。 菅義偉政権発足後、「改革」という言葉がよく聞こえて{\ldots}
4   読売新聞ソウル支局の記者(34)が7月中旬、公務執行妨害の容疑で韓国当局に逮捕されていたこ{\ldots}
5  「おばあちゃんが足立区議の件で怒っていて、手紙を書くらしい」  母からこんなLINEが来たの{\ldots}
6  「『六角家』は破産しましたが、ジャンルとしての『家系ラーメン』は、年々店舗数が拡大しており、{\ldots}
7  《皆さんこんばんは。今日はいつも応援して頂いている皆様にお伝えしたいことがあります。私、佐藤{\ldots}
8  ──本当にあれでいいんだろうか? 帰路、雨の道央道をレンタカーでひた走りながら、そんな思いが{\ldots}
9   掛け布団がだんだん心地よくなってきた、今日この頃。朝夜の急激な気温差に、薄めの上着を羽織っ{\ldots}
\end{Verbatim}
\end{tcolorbox}
        
    \hypertarget{ux30cdux30acux30ddux30b8ux306eux5224ux65adux57faux6e96ux306bux306fux5358ux8a9eux611fux60c5ux6975ux6027ux5bfeux5fdcux8868ux3092ux4f7fux7528ux3057ux307eux3059}{%
\subsubsection{ネガポジの判断基準には「単語感情極性対応表」を使用します。}\label{ux30cdux30acux30ddux30b8ux306eux5224ux65adux57faux6e96ux306bux306fux5358ux8a9eux611fux60c5ux6975ux6027ux5bfeux5fdcux8868ux3092ux4f7fux7528ux3057ux307eux3059}}

この「\href{http://www.lr.pi.titech.ac.jp/~takamura/pndic_ja.html}{単語感情極性対応表}」を分析で使用するための形に整えます。

    \begin{tcolorbox}[breakable, size=fbox, boxrule=1pt, pad at break*=1mm,colback=cellbackground, colframe=cellborder]
\prompt{In}{incolor}{9}{\boxspacing}
\begin{Verbatim}[commandchars=\\\{\}]
\PY{n}{p\PYZus{}dic} \PY{o}{=} \PY{n}{pathlib}\PY{o}{.}\PY{n}{Path}\PY{p}{(}\PY{l+s+s1}{\PYZsq{}}\PY{l+s+s1}{/work/dic}\PY{l+s+s1}{\PYZsq{}}\PY{p}{)} \PY{c+c1}{\PYZsh{} workディレクトリのdicフォルダにパスを通します。ここに「単語感情極性対応表」のファイルを置いています。}

\PY{k}{for} \PY{n}{i} \PY{o+ow}{in} \PY{n}{p\PYZus{}dic}\PY{o}{.}\PY{n}{glob}\PY{p}{(}\PY{l+s+s1}{\PYZsq{}}\PY{l+s+s1}{*.txt}\PY{l+s+s1}{\PYZsq{}}\PY{p}{)}\PY{p}{:} \PY{c+c1}{\PYZsh{} 該当のファイルを見つけます。}
    \PY{k}{with} \PY{n+nb}{open} \PY{p}{(}\PY{n}{i}\PY{p}{,} \PY{l+s+s1}{\PYZsq{}}\PY{l+s+s1}{r}\PY{l+s+s1}{\PYZsq{}}\PY{p}{,} \PY{n}{encoding}\PY{o}{=}\PY{l+s+s1}{\PYZsq{}}\PY{l+s+s1}{utf\PYZhy{}8}\PY{l+s+s1}{\PYZsq{}}\PY{p}{)} \PY{k}{as} \PY{n}{f}\PY{p}{:}
        \PY{n}{x} \PY{o}{=} \PY{p}{[}\PY{n}{i}\PY{o}{.}\PY{n}{replace}\PY{p}{(}\PY{l+s+s1}{\PYZsq{}}\PY{l+s+se}{\PYZbs{}n}\PY{l+s+s1}{\PYZsq{}}\PY{p}{,} \PY{l+s+s1}{\PYZsq{}}\PY{l+s+s1}{\PYZsq{}}\PY{p}{)}\PY{o}{.}\PY{n}{split}\PY{p}{(}\PY{l+s+s1}{\PYZsq{}}\PY{l+s+s1}{:}\PY{l+s+s1}{\PYZsq{}}\PY{p}{)} \PY{k}{for} \PY{n}{i} \PY{o+ow}{in} \PY{n}{f}\PY{o}{.}\PY{n}{readlines}\PY{p}{(}\PY{p}{)}\PY{p}{]} \PY{c+c1}{\PYZsh{} 1行ずつ読み込みます。}
        
\PY{n}{posi\PYZus{}nega\PYZus{}df} \PY{o}{=} \PY{n}{pd}\PY{o}{.}\PY{n}{DataFrame}\PY{p}{(}\PY{n}{x}\PY{p}{,} \PY{n}{columns} \PY{o}{=} \PY{p}{[}\PY{l+s+s1}{\PYZsq{}}\PY{l+s+s1}{基本形}\PY{l+s+s1}{\PYZsq{}}\PY{p}{,} \PY{l+s+s1}{\PYZsq{}}\PY{l+s+s1}{読み}\PY{l+s+s1}{\PYZsq{}}\PY{p}{,} \PY{l+s+s1}{\PYZsq{}}\PY{l+s+s1}{品詞}\PY{l+s+s1}{\PYZsq{}}\PY{p}{,} \PY{l+s+s1}{\PYZsq{}}\PY{l+s+s1}{スコア}\PY{l+s+s1}{\PYZsq{}}\PY{p}{]}\PY{p}{)} \PY{c+c1}{\PYZsh{} 読み込んだデータをDataFrameに格納します。}
\PY{n}{posi\PYZus{}nega\PYZus{}df}\PY{p}{[}\PY{l+s+s1}{\PYZsq{}}\PY{l+s+s1}{読み}\PY{l+s+s1}{\PYZsq{}}\PY{p}{]} \PY{o}{=} \PY{n}{posi\PYZus{}nega\PYZus{}df}\PY{p}{[}\PY{l+s+s1}{\PYZsq{}}\PY{l+s+s1}{読み}\PY{l+s+s1}{\PYZsq{}}\PY{p}{]}\PY{o}{.}\PY{n}{apply}\PY{p}{(}\PY{k}{lambda} \PY{n}{x} \PY{p}{:} \PY{n}{jaconv}\PY{o}{.}\PY{n}{hira2kata}\PY{p}{(}\PY{n}{x}\PY{p}{)}\PY{p}{)} \PY{c+c1}{\PYZsh{} 平仮名をカタカナに変換(同じ読みのものが含まれており、重複を無くす為のようです。)}
\PY{n}{posi\PYZus{}nega\PYZus{}df} \PY{o}{=} \PY{n}{posi\PYZus{}nega\PYZus{}df}\PY{p}{[}\PY{o}{\PYZti{}}\PY{n}{posi\PYZus{}nega\PYZus{}df}\PY{p}{[}\PY{p}{[}\PY{l+s+s1}{\PYZsq{}}\PY{l+s+s1}{基本形}\PY{l+s+s1}{\PYZsq{}}\PY{p}{,} \PY{l+s+s1}{\PYZsq{}}\PY{l+s+s1}{読み}\PY{l+s+s1}{\PYZsq{}}\PY{p}{,} \PY{l+s+s1}{\PYZsq{}}\PY{l+s+s1}{品詞}\PY{l+s+s1}{\PYZsq{}}\PY{p}{]}\PY{p}{]}\PY{o}{.}\PY{n}{duplicated}\PY{p}{(}\PY{p}{)}\PY{p}{]} \PY{c+c1}{\PYZsh{} 重複を削除します。}
\PY{n}{posi\PYZus{}nega\PYZus{}df}\PY{o}{.}\PY{n}{head}\PY{p}{(}\PY{p}{)}
\end{Verbatim}
\end{tcolorbox}

            \begin{tcolorbox}[breakable, size=fbox, boxrule=.5pt, pad at break*=1mm, opacityfill=0]
\prompt{Out}{outcolor}{9}{\boxspacing}
\begin{Verbatim}[commandchars=\\\{\}]
    基本形    読み   品詞       スコア
0   優れる  スグレル   動詞         1
1    良い    ヨイ  形容詞  0.999995
2    喜ぶ  ヨロコブ   動詞  0.999979
3   褒める   ホメル   動詞  0.999979
4  めでたい  メデタイ  形容詞  0.999645
\end{Verbatim}
\end{tcolorbox}
        
    \hypertarget{ux8a18ux4e8bux672cux6587ux3092ux5f62ux614bux7d20ux89e3ux6790ux3057ux3066ux5206ux6790ux306bux5229ux7528ux3067ux304dux308bux5f62ux306bux3057ux307eux3059}{%
\subsubsection{記事本文を形態素解析して分析に利用できる形にします。}\label{ux8a18ux4e8bux672cux6587ux3092ux5f62ux614bux7d20ux89e3ux6790ux3057ux3066ux5206ux6790ux306bux5229ux7528ux3067ux304dux308bux5f62ux306bux3057ux307eux3059}}

形態素解析には\texttt{Tokenizer()}と\texttt{UnicodeNormalizeCharFilter()}を利用します。\\
単語、基本形、品詞、読みを取り出してDataFrameに格納します。\\
そして、記事のDataFrameと「単語感情極性対応表」をマージして、記事に含まれるワードをスコア化します。

    \begin{tcolorbox}[breakable, size=fbox, boxrule=1pt, pad at break*=1mm,colback=cellbackground, colframe=cellborder]
\prompt{In}{incolor}{10}{\boxspacing}
\begin{Verbatim}[commandchars=\\\{\}]
\PY{n}{i} \PY{o}{=} \PY{l+m+mi}{0} \PY{c+c1}{\PYZsh{} このiはニュースNo.を取得する際に利用します。}

\PY{n}{t} \PY{o}{=} \PY{n}{Tokenizer}\PY{p}{(}\PY{p}{)}
\PY{n}{char\PYZus{}filters} \PY{o}{=} \PY{p}{[}\PY{n}{UnicodeNormalizeCharFilter}\PY{p}{(}\PY{p}{)}\PY{p}{]}
\PY{n}{analyzer} \PY{o}{=} \PY{n}{Analyzer}\PY{p}{(}\PY{n}{char\PYZus{}filters}\PY{o}{=}\PY{n}{char\PYZus{}filters}\PY{p}{,} \PY{n}{tokenizer}\PY{o}{=}\PY{n}{t}\PY{p}{)}

\PY{n}{word\PYZus{}lists} \PY{o}{=} \PY{p}{[}\PY{p}{]}
\PY{k}{for} \PY{n}{i}\PY{p}{,} \PY{n}{row} \PY{o+ow}{in} \PY{n}{newslist}\PY{o}{.}\PY{n}{iterrows}\PY{p}{(}\PY{p}{)}\PY{p}{:} \PY{c+c1}{\PYZsh{} iを一つずつ増やしていきニュースNo.とします。}
    \PY{k}{for} \PY{n}{t} \PY{o+ow}{in} \PY{n}{analyzer}\PY{o}{.}\PY{n}{analyze}\PY{p}{(}\PY{n}{row}\PY{p}{[}\PY{l+m+mi}{3}\PY{p}{]}\PY{p}{)}\PY{p}{:} \PY{c+c1}{\PYZsh{} 取り出したレーベルの3カラム目に本文が格納されています。}
        \PY{n}{surf} \PY{o}{=} \PY{n}{t}\PY{o}{.}\PY{n}{surface} \PY{c+c1}{\PYZsh{} 単語}
        \PY{n}{base} \PY{o}{=} \PY{n}{t}\PY{o}{.}\PY{n}{base\PYZus{}form} \PY{c+c1}{\PYZsh{} 基本形}
        \PY{n}{pos} \PY{o}{=} \PY{n}{t}\PY{o}{.}\PY{n}{part\PYZus{}of\PYZus{}speech} \PY{c+c1}{\PYZsh{} 品詞}
        \PY{n}{reading} \PY{o}{=} \PY{n}{t}\PY{o}{.}\PY{n}{reading} \PY{c+c1}{\PYZsh{} 読み}
        
        \PY{n}{word\PYZus{}lists}\PY{o}{.}\PY{n}{append}\PY{p}{(}\PY{p}{[}\PY{n}{i}\PY{p}{,} \PY{n}{surf}\PY{p}{,} \PY{n}{base}\PY{p}{,} \PY{n}{pos}\PY{p}{,} \PY{n}{reading}\PY{p}{]}\PY{p}{)} \PY{c+c1}{\PYZsh{} word\PYZus{}listsに追加}
\PY{n}{word\PYZus{}df} \PY{o}{=} \PY{n}{pd}\PY{o}{.}\PY{n}{DataFrame}\PY{p}{(}\PY{n}{word\PYZus{}lists}\PY{p}{,} \PY{n}{columns}\PY{o}{=}\PY{p}{[}\PY{l+s+s1}{\PYZsq{}}\PY{l+s+s1}{ニュースNo.}\PY{l+s+s1}{\PYZsq{}}\PY{p}{,} \PY{l+s+s1}{\PYZsq{}}\PY{l+s+s1}{単語}\PY{l+s+s1}{\PYZsq{}}\PY{p}{,} \PY{l+s+s1}{\PYZsq{}}\PY{l+s+s1}{基本形}\PY{l+s+s1}{\PYZsq{}}\PY{p}{,} \PY{l+s+s1}{\PYZsq{}}\PY{l+s+s1}{品詞}\PY{l+s+s1}{\PYZsq{}}\PY{p}{,} \PY{l+s+s1}{\PYZsq{}}\PY{l+s+s1}{読み}\PY{l+s+s1}{\PYZsq{}}\PY{p}{]}\PY{p}{)}
\PY{n}{word\PYZus{}df}\PY{p}{[}\PY{l+s+s1}{\PYZsq{}}\PY{l+s+s1}{品詞}\PY{l+s+s1}{\PYZsq{}}\PY{p}{]} \PY{o}{=} \PY{n}{word\PYZus{}df}\PY{p}{[}\PY{l+s+s1}{\PYZsq{}}\PY{l+s+s1}{品詞}\PY{l+s+s1}{\PYZsq{}}\PY{p}{]}\PY{o}{.}\PY{n}{apply}\PY{p}{(}\PY{k}{lambda} \PY{n}{x} \PY{p}{:} \PY{n}{x}\PY{o}{.}\PY{n}{split}\PY{p}{(}\PY{l+s+s1}{\PYZsq{}}\PY{l+s+s1}{,}\PY{l+s+s1}{\PYZsq{}}\PY{p}{)}\PY{p}{[}\PY{l+m+mi}{0}\PY{p}{]}\PY{p}{)} \PY{c+c1}{\PYZsh{} 品詞は複数格納されるが最初の1つのみ利用}
\PY{n}{display}\PY{p}{(}\PY{n}{word\PYZus{}df}\PY{o}{.}\PY{n}{head}\PY{p}{(}\PY{p}{)}\PY{p}{)} \PY{c+c1}{\PYZsh{} 作成した本文のテーブルを表示}
\PY{n+nb}{print}\PY{p}{(}\PY{l+s+s2}{\PYZdq{}}\PY{l+s+s2}{↓↓↓↓↓↓↓単語感情極性対応表とマージ↓↓↓↓↓↓↓}\PY{l+s+s2}{\PYZdq{}}\PY{p}{)}
\PY{n}{score\PYZus{}result} \PY{o}{=} \PY{n}{pd}\PY{o}{.}\PY{n}{merge}\PY{p}{(}\PY{n}{word\PYZus{}df}\PY{p}{,} \PY{n}{posi\PYZus{}nega\PYZus{}df}\PY{p}{,} \PY{n}{on}\PY{o}{=}\PY{p}{[}\PY{l+s+s1}{\PYZsq{}}\PY{l+s+s1}{基本形}\PY{l+s+s1}{\PYZsq{}}\PY{p}{,} \PY{l+s+s1}{\PYZsq{}}\PY{l+s+s1}{品詞}\PY{l+s+s1}{\PYZsq{}}\PY{p}{,} \PY{l+s+s1}{\PYZsq{}}\PY{l+s+s1}{読み}\PY{l+s+s1}{\PYZsq{}}\PY{p}{]}\PY{p}{,} \PY{n}{how}\PY{o}{=}\PY{l+s+s1}{\PYZsq{}}\PY{l+s+s1}{left}\PY{l+s+s1}{\PYZsq{}}\PY{p}{)} \PY{c+c1}{\PYZsh{} 本文のテーブルと単語感情極性対応表をマージ}
\PY{n}{display}\PY{p}{(}\PY{n}{score\PYZus{}result}\PY{o}{.}\PY{n}{head}\PY{p}{(}\PY{p}{)}\PY{p}{)} \PY{c+c1}{\PYZsh{} 作成したスコアテーブルを表示}
\end{Verbatim}
\end{tcolorbox}

    
    \begin{verbatim}
   ニュースNo.   単語  基本形  品詞   読み
0        0   ジャ   ジャ  名詞   ジャ
1        0  ニーズ  ニーズ  名詞  ニーズ
2        0   Jr   Jr  名詞    *
3        0    .    .  名詞    *
4        0    内    内  名詞   ナイ
    \end{verbatim}

    
    \begin{Verbatim}[commandchars=\\\{\}]
↓↓↓↓↓↓↓単語感情極性対応表とマージ↓↓↓↓↓↓↓
    \end{Verbatim}

    
    \begin{verbatim}
   ニュースNo.   単語  基本形  品詞   読み        スコア
0        0   ジャ   ジャ  名詞   ジャ        NaN
1        0  ニーズ  ニーズ  名詞  ニーズ  -0.163536
2        0   Jr   Jr  名詞    *        NaN
3        0    .    .  名詞    *        NaN
4        0    内    内  名詞   ナイ   -0.74522
    \end{verbatim}

    
    \begin{tcolorbox}[breakable, size=fbox, boxrule=1pt, pad at break*=1mm,colback=cellbackground, colframe=cellborder]
\prompt{In}{incolor}{11}{\boxspacing}
\begin{Verbatim}[commandchars=\\\{\}]
\PY{n}{result} \PY{o}{=} \PY{p}{[}\PY{p}{]}
\PY{k}{for} \PY{n}{i} \PY{o+ow}{in} \PY{n+nb}{range}\PY{p}{(}\PY{n+nb}{len}\PY{p}{(}\PY{n}{score\PYZus{}result}\PY{p}{[}\PY{l+s+s1}{\PYZsq{}}\PY{l+s+s1}{ニュースNo.}\PY{l+s+s1}{\PYZsq{}}\PY{p}{]}\PY{o}{.}\PY{n}{unique}\PY{p}{(}\PY{p}{)}\PY{p}{)}\PY{p}{)}\PY{p}{:} \PY{c+c1}{\PYZsh{} ニュースNo.を利用してfor文を回します。}
    \PY{n}{temp\PYZus{}df} \PY{o}{=} \PY{n}{score\PYZus{}result}\PY{p}{[}\PY{n}{score\PYZus{}result}\PY{p}{[}\PY{l+s+s1}{\PYZsq{}}\PY{l+s+s1}{ニュースNo.}\PY{l+s+s1}{\PYZsq{}}\PY{p}{]}\PY{o}{==} \PY{n}{i}\PY{p}{]}
    \PY{n}{text} \PY{o}{=} \PY{l+s+s1}{\PYZsq{}}\PY{l+s+s1}{\PYZsq{}}\PY{o}{.}\PY{n}{join}\PY{p}{(}\PY{n+nb}{list}\PY{p}{(}\PY{n}{temp\PYZus{}df}\PY{p}{[}\PY{l+s+s1}{\PYZsq{}}\PY{l+s+s1}{単語}\PY{l+s+s1}{\PYZsq{}}\PY{p}{]}\PY{p}{)}\PY{p}{)} \PY{c+c1}{\PYZsh{} 1タイトル無いの全ての単語をつなげる。}
    \PY{n}{score} \PY{o}{=} \PY{n}{temp\PYZus{}df}\PY{p}{[}\PY{l+s+s1}{\PYZsq{}}\PY{l+s+s1}{スコア}\PY{l+s+s1}{\PYZsq{}}\PY{p}{]}\PY{o}{.}\PY{n}{astype}\PY{p}{(}\PY{n+nb}{float}\PY{p}{)}\PY{o}{.}\PY{n}{sum}\PY{p}{(}\PY{p}{)} \PY{c+c1}{\PYZsh{} 1タイトル内のスコアを全て足し合わせる。➡︎累計スコア}
    \PY{n}{score\PYZus{}r} \PY{o}{=} \PY{n}{score}\PY{o}{/}\PY{n}{temp\PYZus{}df}\PY{p}{[}\PY{l+s+s1}{\PYZsq{}}\PY{l+s+s1}{スコア}\PY{l+s+s1}{\PYZsq{}}\PY{p}{]}\PY{o}{.}\PY{n}{astype}\PY{p}{(}\PY{n+nb}{float}\PY{p}{)}\PY{o}{.}\PY{n}{count}\PY{p}{(}\PY{p}{)} \PY{c+c1}{\PYZsh{} 本文の長さに影響されないように単語数で割り算する。➡︎標準化スコア}
    \PY{n}{result}\PY{o}{.}\PY{n}{append}\PY{p}{(}\PY{p}{[}\PY{n}{i}\PY{p}{,} \PY{n}{text}\PY{p}{,} \PY{n}{score}\PY{p}{,} \PY{n}{score\PYZus{}r}\PY{p}{]}\PY{p}{)}

\PY{n}{ranking} \PY{o}{=} \PY{n}{pd}\PY{o}{.}\PY{n}{DataFrame}\PY{p}{(}\PY{n}{result}\PY{p}{,} \PY{n}{columns}\PY{o}{=}\PY{p}{[}\PY{l+s+s1}{\PYZsq{}}\PY{l+s+s1}{ニュースNo.}\PY{l+s+s1}{\PYZsq{}}\PY{p}{,} \PY{l+s+s1}{\PYZsq{}}\PY{l+s+s1}{テキスト}\PY{l+s+s1}{\PYZsq{}}\PY{p}{,} \PY{l+s+s1}{\PYZsq{}}\PY{l+s+s1}{累計スコア}\PY{l+s+s1}{\PYZsq{}}\PY{p}{,} \PY{l+s+s1}{\PYZsq{}}\PY{l+s+s1}{標準化スコア}\PY{l+s+s1}{\PYZsq{}}\PY{p}{]}\PY{p}{)}\PY{o}{.}\PY{n}{sort\PYZus{}values}\PY{p}{(}\PY{n}{by}\PY{o}{=}\PY{l+s+s1}{\PYZsq{}}\PY{l+s+s1}{標準化スコア}\PY{l+s+s1}{\PYZsq{}}\PY{p}{)}\PY{o}{.}\PY{n}{reset\PYZus{}index}\PY{p}{(}\PY{n}{drop}\PY{o}{=}\PY{k+kc}{True}\PY{p}{)} \PY{c+c1}{\PYZsh{} 標準化スコアで並び替えてDataFrameに格納}
\end{Verbatim}
\end{tcolorbox}

    \begin{tcolorbox}[breakable, size=fbox, boxrule=1pt, pad at break*=1mm,colback=cellbackground, colframe=cellborder]
\prompt{In}{incolor}{12}{\boxspacing}
\begin{Verbatim}[commandchars=\\\{\}]
\PY{n}{ranking} \PY{o}{=} \PY{n}{pd}\PY{o}{.}\PY{n}{merge}\PY{p}{(}\PY{n}{ranking}\PY{p}{,} \PY{n}{newslist}\PY{p}{[}\PY{p}{[}\PY{l+s+s1}{\PYZsq{}}\PY{l+s+s1}{ニュースNo.}\PY{l+s+s1}{\PYZsq{}}\PY{p}{,} \PY{l+s+s1}{\PYZsq{}}\PY{l+s+s1}{title}\PY{l+s+s1}{\PYZsq{}}\PY{p}{,} \PY{l+s+s1}{\PYZsq{}}\PY{l+s+s1}{url}\PY{l+s+s1}{\PYZsq{}}\PY{p}{]}\PY{p}{]}\PY{p}{,} \PY{n}{on}\PY{o}{=}\PY{l+s+s1}{\PYZsq{}}\PY{l+s+s1}{ニュースNo.}\PY{l+s+s1}{\PYZsq{}}\PY{p}{,} \PY{n}{how}\PY{o}{=}\PY{l+s+s1}{\PYZsq{}}\PY{l+s+s1}{left}\PY{l+s+s1}{\PYZsq{}}\PY{p}{)} \PY{c+c1}{\PYZsh{} ニュースNo.基準でマージする。タイトルとURLを追加する。}
\PY{n}{ranking} \PY{o}{=} \PY{n}{ranking}\PY{o}{.}\PY{n}{reindex}\PY{p}{(}\PY{n}{columns}\PY{o}{=}\PY{p}{[}\PY{l+s+s1}{\PYZsq{}}\PY{l+s+s1}{ニュースNo.}\PY{l+s+s1}{\PYZsq{}}\PY{p}{,} \PY{l+s+s1}{\PYZsq{}}\PY{l+s+s1}{title}\PY{l+s+s1}{\PYZsq{}}\PY{p}{,} \PY{l+s+s1}{\PYZsq{}}\PY{l+s+s1}{url}\PY{l+s+s1}{\PYZsq{}}\PY{p}{,} \PY{l+s+s1}{\PYZsq{}}\PY{l+s+s1}{テキスト}\PY{l+s+s1}{\PYZsq{}}\PY{p}{,} \PY{l+s+s1}{\PYZsq{}}\PY{l+s+s1}{累計スコア}\PY{l+s+s1}{\PYZsq{}}\PY{p}{,} \PY{l+s+s1}{\PYZsq{}}\PY{l+s+s1}{標準化スコア}\PY{l+s+s1}{\PYZsq{}}\PY{p}{]}\PY{p}{)} \PY{c+c1}{\PYZsh{} カラムを並び替え}
\end{Verbatim}
\end{tcolorbox}

    \begin{tcolorbox}[breakable, size=fbox, boxrule=1pt, pad at break*=1mm,colback=cellbackground, colframe=cellborder]
\prompt{In}{incolor}{13}{\boxspacing}
\begin{Verbatim}[commandchars=\\\{\}]
\PY{n}{ranking}
\end{Verbatim}
\end{tcolorbox}

            \begin{tcolorbox}[breakable, size=fbox, boxrule=.5pt, pad at break*=1mm, opacityfill=0]
\prompt{Out}{outcolor}{13}{\boxspacing}
\begin{Verbatim}[commandchars=\\\{\}]
   ニュースNo.                                              title  \textbackslash{}
0        2                 「365日24時間働こう」……ワタミの“思想教育”はいまも続いていた
1        4                 警察官に唾吐き公務執行妨害で逮捕 読売新聞がソウル支局記者を懲戒処分
2        3        “縦割り行政”の弊害だ! 橋下徹が語る「Go Toキャンペーンは何が間違っているのか」
3        0  ジャニーズ元MADEリーダー・稲葉光(29)が元Berryz工房アイドルと“渋谷ホテルデート{\ldots}
4        8        安倍政権最大の功績は“アイヌ博物館”だった? 200億円をブチ込んだ「ウポポイ」の虚実
5        9               「長さ」ではなく…美容師が教える、髪を切るときに言うといい「意外な言葉」
6        5      同性愛差別の足立区議宛てに「とんでもない思い違いです」…81歳祖母の手紙が教えてくれたこと
7        1                    2050年「CO2排出量ゼロ」を宣言したJR東日本の戦略とは?
8        7  《平手派また卒業》欅坂46・佐藤詩織が書いた「活動のなかで悲しかったこと」 櫻坂46の“急勾{\ldots}
9        6        家系ラーメン“のれん分け戦争”「吉村家vs.六角家」裏切りと屈服の黒歴史〈六角家破産〉

                                    url  \textbackslash{}
0  https://bunshun.jp//articles/-/40843
1  https://bunshun.jp//articles/-/40868
2  https://bunshun.jp//articles/-/40877
3  https://bunshun.jp//articles/-/40869
4  https://bunshun.jp//articles/-/40841
5  https://bunshun.jp//articles/-/40694
6  https://bunshun.jp//articles/-/40826
7  https://bunshun.jp//articles/-/40083
8  https://bunshun.jp//articles/-/40862
9  https://bunshun.jp//articles/-/40752

                                                テキスト       累計スコア    標準化スコア
0  ワタミ株式会社の労働問題に関する告発が続いている。10月2日、「ワタミの宅食」営業所の所長が{\ldots} -322.836678 -0.530981
1  読売新聞ソウル支局の記者(34)が7月中旬、公務執行妨害の容疑で韓国当局に逮捕されていたこと{\ldots} -144.916148 -0.489582
2  「規制改革」「行政改革」「縦割り打破」。 菅義偉政権発足後、「改革」という言葉がよく聞こえて{\ldots}  -94.718989 -0.480807
3  ジャニーズJr.内の人気ユニット「宇宙Six」の山本亮太(30)が違法な闇スロット店に通って{\ldots} -196.888853 -0.479048
4  ──本当にあれでいいんだろうか? 帰路、雨の道央道をレンタカーでひた走りながら、そんな思いが{\ldots} -483.393151 -0.476719
5  掛け布団がだんだん心地よくなってきた、今日この頃。朝夜の急激な気温差に、薄めの上着を羽織った{\ldots} -192.702889 -0.475810
6  「おばあちゃんが足立区議の件で怒っていて、手紙を書くらしい」  母からこんなLINEが来たの{\ldots} -213.244051 -0.460570
7  地球温暖化に歯止めをかけるため、一日も早い脱炭素社会の実現を。JR東日本では今年5月、205{\ldots} -197.744554 -0.457742
8  《皆さんこんばんは。今日はいつも応援して頂いている皆様にお伝えしたいことがあります。私、佐藤{\ldots} -136.887378 -0.447344
9  「『六角家』は破産しましたが、ジャンルとしての『家系ラーメン』は、年々店舗数が拡大しており、{\ldots} -238.565564 -0.409203
\end{Verbatim}
\end{tcolorbox}
        
    \begin{tcolorbox}[breakable, size=fbox, boxrule=1pt, pad at break*=1mm,colback=cellbackground, colframe=cellborder]
\prompt{In}{incolor}{14}{\boxspacing}
\begin{Verbatim}[commandchars=\\\{\}]
\PY{n+nb}{print}\PY{p}{(}\PY{l+s+s2}{\PYZdq{}}\PY{l+s+s2}{\PYZlt{}\PYZlt{}ポジティブ1位\PYZgt{}\PYZgt{}}\PY{l+s+s2}{\PYZdq{}}\PY{p}{,} \PY{n}{end}\PY{o}{=}\PY{l+s+s2}{\PYZdq{}}\PY{l+s+se}{\PYZbs{}n}\PY{l+s+se}{\PYZbs{}n}\PY{l+s+s2}{\PYZdq{}}\PY{p}{)}
\PY{n+nb}{print}\PY{p}{(}\PY{n}{ranking}\PY{o}{.}\PY{n}{iloc}\PY{p}{[}\PY{o}{\PYZhy{}}\PY{l+m+mi}{1}\PY{p}{,} \PY{l+m+mi}{1}\PY{p}{]}\PY{p}{)}
\PY{n+nb}{print}\PY{p}{(}\PY{n}{ranking}\PY{o}{.}\PY{n}{iloc}\PY{p}{[}\PY{o}{\PYZhy{}}\PY{l+m+mi}{1}\PY{p}{,} \PY{l+m+mi}{2}\PY{p}{]}\PY{p}{)}
\PY{n+nb}{print}\PY{p}{(}\PY{n}{ranking}\PY{o}{.}\PY{n}{iloc}\PY{p}{[}\PY{o}{\PYZhy{}}\PY{l+m+mi}{1}\PY{p}{,} \PY{l+m+mi}{3}\PY{p}{]}\PY{p}{)}
\end{Verbatim}
\end{tcolorbox}

    \begin{Verbatim}[commandchars=\\\{\}]
<<ポジティブ1位>>

家系ラーメン“のれん分け戦争”「吉村家vs.六角家」裏切りと屈服の黒歴史〈六角家破産〉
https://bunshun.jp//articles/-/40752
「『六角家』は破産しましたが、ジャンルとしての『家系ラーメン』は、年々店舗数が拡大しており、今が最盛期です。その背景には“資本系”の台頭があります。でもね、確か
に資本系も美味しいんですが、かつて『六角家』や『吉村家』の職人たちが日々、味の改良に挑み、切磋琢磨して、互いに覇を競っていたあの時代が、倒産のニュースを聞いた今
、妙に懐かしく輝いて見えるんです」(ラーメン評論家・山本剛志氏)(前編「家系ラーメン・名門『六角家』はなぜ破産したか」を読む) 2020年9月4日、かつて家系ラ
ーメンの「代名詞」とまで呼ばれ、一世を風靡した老舗の名店「六角家」が、横浜地裁より破産手続き開始決定を受けていると報じられた。2017年に「六角家」の本店が閉じ
られ、倒産は時間の問題だったとみるファンも多かったが、いざこのニュースが流れると「マジか」「もう一度食べたかった」など、その倒産を惜しむ声がネット上などで多く上
がった。そして、冒頭の山本氏のように、「六角家」がこれまで「家系ラーメン」業界の隆盛にはたした役割について、再評価する声も上がってきている。「そもそも、家系ラー
メンが誕生したのは1974年。長距離トラックの運転手だった吉村実氏が仕事の合間に趣味で密かに研究を重ねていた九州の豚骨ベースと東京の醤油ベースを組み合わせたスー
プを開発したことに端を発しています。この味ならいける! と判断した吉村氏はその後会社を辞め、『吉村家』というラーメン店を横浜の磯子区新杉田に開店。豚骨醤油ベース
の風味豊かなスープと、モッチリとした『酒井製麺』の太麺を合わせた1杯は、本人の読みどおり大きな人気を集め、店は日々、大勢のお客さんで賑わいました。 評判を集めた
「吉村家」には、多くの才能が集まりました。そして、吉村氏の元で修業をした弟子たちは、吉村氏直々に教わり技術を身に着けると“暖簾分け”という形で自分の店を持つよう
になり、弟子たちの店もそれぞれのエリアで大人気店となりました。独立した弟子たちが店名に『\textasciitilde{}家』とつけることが多かったため、ファンの間で、『吉村家』とそこから独立
した店は『家系』と呼ばれるようになりました。現在は、“家系皆伝”という証書を吉村氏からもらった店舗のみが吉村家の“直系店舗”として認められています」(ラーメン探
究家・田中一明氏) 直系店舗にはいくつか特徴がある。まずは、麺は、酒井製麺所という製麺所から卸している麺でなければならない。『家系といえば酒井製麺』と言われるほ
どその結びつきは固く、酒井製麺の麺は中太でモチモチした独特の食感があるという。また、麺は寸胴で茹で、テボザルではなく平ザルですくい上げるのも特徴だ。 その直系店
鋪以外で、酒井製麺所の麺を使っていたのが今回倒産した「六角家」と、「本牧家」だった。「『六角家』は、吉村氏の一番弟子である神藤隆氏が独立して横浜市神奈川区六角橋
で始めた店です。一時は本家『吉村家』や吉村氏が横浜市中区に出したのれん分け1号店である『本牧家』とともに家系御三家と言われたほどの超有名店です。ただ、『六角家』
は吉村氏の一番弟子が始めた店にもかかわらず、開店当初から今日まで、一度も吉村氏から吉村家の“直系店舗”として認められたことはありません。『吉村家』と『六角家』は
ずっとぎくしゃくした関係が続いていたのです」(食ジャーナリスト・小林孝充氏) 一体なぜなのか。ラーメン探究家の田中一明氏が解説する。「吉村氏は1974年に横浜市
磯子区新杉田に『吉村家』1号店を開いた後、12年間は支店をつくりませんでしたが、1986年満を持して横浜市中区本牧に2号店である『本牧家』をオープンさせました。
この店を弟子の神藤隆氏に任せ、吉村氏はオーナーとして、いよいよ店舗の拡大へと乗り出したかのように見えました。ところがその2年後、店主の神藤氏が従業員を引き連れ、
本牧家を出て行ってしまったのです。さらにその後、神藤氏は神奈川区六角橋に自ら『六角家』を開店しました。一方、神藤氏という主を失った『本牧家』は一時期休業を余儀な
くされてしまったのです。 なぜ神藤氏は独立したのかについては、当時様々な憶測が飛び交いましたが、私は目指す味の方向性が違っていたことが大きな理由だと思っています
。『吉村家』が醤油ダレをガツンと効かせたパンチのあるタイプであるのに対し、『六角家』は豚骨と醤油ダレをバランス良く効かせた食べ手を選ばないタイプであるのが特徴。
神藤さんは自ら信じる味を追求したかったということではないでしょうか」 「吉村家」と袂を分かって誕生した「六角家」は、独自路線で成長していった。「六角家」の転機に
なったのは1994年、新横浜駅近くに誕生した「新横浜ラーメン博物館」に横浜代表として出店したことだった。これにより、全国的な知名度を得た「六角家」は本家の「吉村
家」や、吉村氏とは別の経営者を迎えた「本牧家」とともに「家系御三家」とファンの間で呼ばれるようになる。この三店は、次々弟子をとり、彼らを独立させることで、直系店
を増やしていった。今でも、この時期の御三家による「争い」はファンの間で語り草となっている。 特に有名なのは、94年に起きた「環2ラーメン戦争」だろう。事の発端は
94年、「本牧家」が環状2号線に面した土地に移転したことに始まる。移転から1年も経たずして、「本牧家」の店舗からわずか徒歩1分の位置に、吉村「吉村家」の直系店舗
「環2家」がオープンしたのだ。「ちょうどこの頃は横浜各地に吉村家の直系店舗が広がり始めた時期。環状2号線はもともとロードサイドのラーメン激戦区でしたから、偶然、
出店が近くになってしまうことはありうる。しかし、『本牧家』と『環2家』はそれにしても近すぎる距離でした。家系ファンの間では『吉村家が直弟子を使って、本牧家をつぶ
しにきた』と囁かれることもありました」(前出・山本氏) しかし、この戦争は「本牧家」と「環2家」両方が客を離さぬために、職人たちが味の改良を進めたという側面もあ
る。「環2ラーメン戦争」の決着はいまだについておらず、今も両店は行列の絶えない人気店であり、徒歩1分の距離で互いに火花を散らしている。 2000年代初頭になって
、吉村家は「杉田家」(横浜市磯子区)、「はじめ家」(富山県魚津市)、「王道家」(千葉県柏市)、「横横家」(横浜市金沢区)などの直系店舗を次々と増やしていった。「
『六角家』の多店舗展開や、『吉村家』で修業した人たちの独立開業によって、家系ラーメンの認知度は上がっていきました。1990年代に入ると、『吉村家』や『六角家』の
出身者が出した店で修業し、独立開業する者が目立つようになりました。『吉村家』、『六角家』からすれば、孫、曾孫にあたる店になりますね。やがて、家系ラーメンを出す店
舗は、発祥の地である横浜を飛び出し、全国各地に広がっていきました。大手飲食企業が、家系ラーメンの人気に商機を見出したのもこの頃だと思います。2010年代に入ると
、いよいよ、それらの企業が業界に本格参入してきます。これらの大手飲食企業が各店舗で提供するラーメンが、いわゆる“資本系”と呼ばれるものです」(前出・田中氏) 資
本系が徐々に勢力を伸ばす一方、職人たちの技に支えられる「家系御三家」の系列のラーメン店同士でも、競争が激化していった。その最たるものは2013年に起きた「三つ巴
の乱戦」(山本氏)である。「『吉村家』の直系店『末廣家』と『六角家』がある県道12号沿いに、元は『吉村家』直系店だった『王道家』の系列である『とらきち家』が開店
したのです。『とらきち家』と『六角家』はわずか2軒隣りの距離。ラーメン店の激戦区として知られている地域ですが、『環2ラーメン戦争』を知っている人たちからしたら、
またかと思わざるを得ない出来事でした。 もう家系ファンたちは沸き上がりましてね。そもそも『末廣家』が2012年に開店したときに『吉村家がとうとう六角家を潰しに来
た』と話題になったのに、その翌年には王道家の系列店がすぐそばに建ったわけです。『吉村家と王道家がタッグを組んで六角家を標的にしたのか』という憶測まで飛び交ったほ
どです」(同前) しかし、この「三つ巴の乱戦」は4年後に「六角家」の店主である神藤氏が体調不良で店に出ることが困難になり、同店を閉めたことで唐突に決着がついた。
「もともと『吉村家』という元祖から派生していった『家系』には、細かな系図があります。あの店は『吉村家』の直系、こっちは『六角家』系列と、ファンは味とともに、“出
自”もすごく気にしていました。お店側も、職人のプライドを持って、俺たちの『家系』を見せてやるという気概に溢れて、日々味の改良に明け暮れていた。たしかに『戦争』と
呼ばれるほど、競争が激化した地域もありましたけど、それもまた味の新陳代謝を生むプラスの側面があった。あれは家系が進化するための『スパイス』になった部分があるんで
す」(同前) かつては「家系」の代名詞ともなり、「吉村家」らと“頂上決戦”を行った「六角家」本店の跡地は雑草が伸び放題の空き地になっている。一方の吉村家は、現在
は家系総本山だと看板に掲げており、その勢いはますます盛んだ。開店から50年近く経つ現在も長蛇の列ができる人気店である。 ラーメン職人たちの戦いと資本系の台頭。「
六角家」の倒産は、家系ラーメンの一つの時代の終焉を意味しているのである。(前編「家系ラーメン・名門『六角家』はなぜ破産したか」を読む)この記事の写真(14枚)
    \end{Verbatim}

    \begin{tcolorbox}[breakable, size=fbox, boxrule=1pt, pad at break*=1mm,colback=cellbackground, colframe=cellborder]
\prompt{In}{incolor}{15}{\boxspacing}
\begin{Verbatim}[commandchars=\\\{\}]
\PY{n+nb}{print}\PY{p}{(}\PY{l+s+s2}{\PYZdq{}}\PY{l+s+s2}{\PYZlt{}\PYZlt{}ネガティブ1位\PYZgt{}\PYZgt{}}\PY{l+s+s2}{\PYZdq{}}\PY{p}{,} \PY{n}{end}\PY{o}{=}\PY{l+s+s2}{\PYZdq{}}\PY{l+s+se}{\PYZbs{}n}\PY{l+s+se}{\PYZbs{}n}\PY{l+s+s2}{\PYZdq{}}\PY{p}{)}
\PY{n+nb}{print}\PY{p}{(}\PY{n}{ranking}\PY{o}{.}\PY{n}{iloc}\PY{p}{[}\PY{l+m+mi}{0}\PY{p}{,} \PY{l+m+mi}{1}\PY{p}{]}\PY{p}{)}
\PY{n+nb}{print}\PY{p}{(}\PY{n}{ranking}\PY{o}{.}\PY{n}{iloc}\PY{p}{[}\PY{l+m+mi}{0}\PY{p}{,} \PY{l+m+mi}{2}\PY{p}{]}\PY{p}{)}
\PY{n+nb}{print}\PY{p}{(}\PY{n}{ranking}\PY{o}{.}\PY{n}{iloc}\PY{p}{[}\PY{l+m+mi}{0}\PY{p}{,} \PY{l+m+mi}{3}\PY{p}{]}\PY{p}{)}
\end{Verbatim}
\end{tcolorbox}

    \begin{Verbatim}[commandchars=\\\{\}]
<<ネガティブ1位>>

「365日24時間働こう」……ワタミの“思想教育”はいまも続いていた
https://bunshun.jp//articles/-/40843
ワタミ株式会社の労働問題に関する告発が続いている。10月2日、「ワタミの宅食」営業所の所長が、労働基準監督署からの残業代未払いの是正勧告、月175時間を超える長
時間労働、上司によるタイムカードの改ざんを次々と公表したのだ。「ホワイト企業」宣伝のワタミで月175時間の残業
残業代未払いで労基署から是正勧告ワタミがホワイト企業になれなかった理由は? 勝手に勤怠「改ざん」システムも Aさんは長時間労働の末、昼夜の感覚がなくなり、「この
まま寝たら、もう目が覚めないのではないか」と恐怖を抱きながら生活するほどだった。「あのまま働いていたら、死んでいた」とAさんは断言する。現在は、精神疾患を発症し
、労災申請をしながら休業中だ。 しかし、Aさんは命の危険を感じていながら、なぜ過酷な仕事を続けてしまったのだろうか。その背景には、労働者の意識に働きかけ、過酷な
労働を受け入れさせてしまう、ワタミによる「思想教育」のシステムがあった。「こんなにいい仕事をしているんだから、苦しくても頑張ろう」「苦しいことも、苦しくない。む
しろ自分の力になる」 Aさんは過重労働の最中、そう自分に言い聞かせていた。 実際、Aさんはワタミの宅食の仕事に「誇り」を感じていたという。確かにワタミの「宅食」
に「社会貢献」と言える部分がある。ワタミの宅食には、食事の調達が難しい高齢者に、定期的に安く食事を届けるというコンセプトがある。Aさんの営業所の届け先も、そうし
た人たちばかりだった。 隔日でデイサービスに通い、隔日で家にいるときの食事に困る一人暮らしの高齢者。毎日昼食用と夕食用の2食を頼んで、ワタミの宅食しか食べていな
い高齢者、障がい者や、親を介護している人{\ldots}。 苦しい生活の事情によって、食事に手間や時間、お金をかけられず、ワタミに頼らざるを得ない様々な利用者がいた
。それまで介護や教育現場で働いてきたAさんにとって、社会や地域のためになる宅食の仕事は、非常にやりがいの感じられるものだった。 だがそれらの「支援」を通じて無理
やり利益をあげようとすれば、長時間労働や低賃金による労働者の犠牲によって支えられることになってしまう。いわば「貧困・ブラック企業ビジネス」とでもいえようか。
Aさんも当初、仕事のやりがいと過酷な労働とは分けるべきものだと冷静に考え、業務のつらさについてはワタミに不満を抱いていた。 しかし、劣悪な労働条件は、Aさんの頭
の中ではいつしか問題とはならなくなっていた。その変化をもたらしたのが、ワタミによる「思想教育」だったと、Aさんは振り返っている。「365日24時間、死ぬまで働け
」 この言葉をご存知の方も多いだろう。渡邉美樹氏の過去の30年以上に渡る文章を抜粋して編集した400ページにも及ぶ「理念集」という書物に記載されていた言葉である
。Aさんは入社直後に会社から「理念集」を渡され、肌身離さず持っているよう言われた。 現在は批判を受けて、「死ぬまで働け」などの極端な表現は削除されている。それで
もAさんが手渡された「理念集」(2016年度版)に削除されずに残っていた、印象的な表現の一部を引用しよう。〈私は、仕事はお金を得る為だけの手段とは考えていません
。仕事とはその人の「生き方」そのものであり、「自己実現の唯一の手段」であると信じています。だから新卒のセミナーでも、時代遅れはなはだしいと言われつつ、「365日
24時間働こう」と言うのです〉〈私も会社説明会で、「仕事とは生きることそのものなんだ。仕事を、お金を得るための手段なんかにしてはいけないんだ。仕事を通して人間性
を高めよう」と訴えています〉〈ワタミタクショクの最大の商品は「人」と知れ。まごころさん(注:宅食の配達員)とは、お弁当に「心」を乗せて運び、「ありがとう」をいた
だく仕事をする人のこと。考え違いをして「お弁当」を運び「お金」をいただく仕事と思う人が出てこないことを祈る〉 このように、同書には、客からの「ありがとう」という
感謝のために働くことを奨励して、賃金のために働くことを批判し、ワタミの利益のために自分を犠牲にすることを正当化する内容が、随所に記されていた。 ワタミに入社した
Aさんは、何かトラブルがあると、支社長から「理念集の第×章をちゃんと読んだ?」と問われ、渡邉美樹氏の「思想」を十分に理解していないせいだと注意されていた。 さら
に、エリアマネージャーによる毎月のカウンセリングがあり、そこでは当月の「社内報」(毎月発行され、理念集の文章の多くがここからの抜粋である)の感想と、理念集の任意
の箇所の感想を書かされるようになっていた。
加えて、4ヶ月に1回のレポートが課せられた。理念集から指定された章の範囲と、直近4ヶ月分の社内報に掲載される渡邉氏直筆の言葉に対する二つの感想が強制されていた。
Aさんはもともと、これらの感想文提出を「気持ち悪い」と感じていた。しかし、嫌々ながら、継続的に書かされているうちに「どこかで渡邉美樹さんの考えを植え付けられてい
く感じがあった」という。 それをさらに深化させたのが、「ビデオレター」の存在だった。 毎月、営業所には渡邉美樹氏が出演する「ビデオレター」が提供された。人気テレ
ビ番組「情熱大陸」のナレーターが起用され、渡邉美樹氏が毎回出演し、ワタミの事業の素晴らしさを説く30分間の映像である。 この映像についても、毎月の感想が義務付け
られていた。しかも書くのは所長だけではない。個人事業主であるはずの配達員までも、毎月ビデオレターを視聴して感想を書くことが契約に盛り込まれていた。 配達員たちは
、営業所でこの映像を見せられると、所長と配達員の名前が羅列されたシートの自分の感想欄(60字ほどは書けるスペースがある)に、手書きで感想を書かされる。なお、配達
員は配達先1軒あたりの報酬が百数十円しかないが、この映像視聴や感想を書くことによる新たな報酬は一切ない。 さらに配達員の感想に、ワタミに対する批判や仕事への不満
などがある場合は、所長がその箇所に印をして、コメントを書き加えるようにと指示されていた。Aさんは独自の工夫として、すべての配達員の感想に対して、その配達員の感想
を超えるほどの字数で、丁寧にコメントをするようにした。 そして、この「コメント」こそが、Aさんの「思想形成」に大きな役割を果たしていたという。 Aさんは、上司の
エリアマネージャーから、「配達員たちに、ワタミを称賛するような感想を書くよう“もっていく”ことも所長の仕事だ」と言われていた。配達員の感想を「コントロール」する
ようにというのだ。 とはいえ、Aさんは配達員の感想の「改ざん」をしたわけではない。まず、Aさんは所長として、配達員に先立って自分の感想欄で、ワタミの事業やワタミ
で働くことの素晴らしさを説いた。すると、後から書く配達員たちは、おのずと所長の「模範解答」を意識して、否定的な感想を書きづらくなる。 それでも配達員の感想に、ワ
タミに疑問を呈するような箇所でもあれば、コメント欄でそこを指摘し、「意義を改めて私と共有しましょう」などと食い下がった。やがて、数ヶ月をかけて配達員たちのコメン
トから否定的な文章は消え、少なくとも表向きは、ワタミを褒め称える感想一色になっていった。 ワタミの素晴らしさをひたすら繰り返す、この毎月のコメントは「思想形成」
に着実に影響をもたらした。ただし、本当に変わったのは、配達員ではなく、所長であるAさんの方だったのである。 ワタミに対する「称賛」を業務の必要上、あえて続けてい
たはずのAさんは、徐々にワタミの事業や労働を無条件に賛美し、労働問題の不満を感じなくなる意識が本当に芽生えてきたという。配達員の感想を何度も指導するという行為を
通じて、Aさん自身の意識こそが「教育」されてしまったのだ。 ここで、Aさんが配達員に見せるために書いた、ビデオレターの感想文を引用してみよう。ワタミのSDGsの
取り組みや、ワタミが行っているカンボジアの学校支援の取り組みの映像をみた感想だ。〈SDGsという言葉が社会に広く知れわたる前からワタミは取り組みを継続してきまし
た。私はそのような会社で仕事させていただき、本当に感謝致します。自分のすぐ近くにも、社会の役に立つことができることは多くあります。いつも勇気と誇りを胸に笑顔を持
って努めて参ります〉〈自分もキラキラした笑顔、そしてまごころさんがキラキラした笑顔でいられる様、前進していきます〉〈ここ(注:Aさんの営業所)を輝かせることこそ
私の仕事です。精一杯、努めて参ります!!〉 ワタミの事業に対する絶賛と、ワタミで働かせてもらっていることの感謝。スペースをはみ出るほどの分量と、異様なほどに高揚
した表現。当時のAさんが本当にワタミを「信奉」する気持ちがなければ、ここまで書くことは難しいだろう。
現在、精神疾患で休業中のAさんは、この自分の感想文を改めて見返し、こう呟いた。「気持ち悪いですね、いま読むと」 月の残業時間が150時間を超えたころ、Aさんは長
時間労働について「私が悪い」と思い詰めていた。多すぎる業務量はワタミの責任なのに、そこは「思想教育」のために受け入れてしまい、むしろ仕事が遅いせいだと自分を責め
るようになっていたのだ。
そんなとき、新しく入った配達員の一人が、深夜まで働くAさんを見て心配し、単刀直入に指摘してくれた。「Aさん、『洗脳』されてるんじゃないの?」
当初、「失礼な人だ」と憤ったが、何度もこの配達員が親身になって指摘してくれるうちに、「私、おかしいのかも」と思い始めるようになっていた。 また、単身赴任中だった
配偶者が、コロナ禍によってテレワークで自宅勤務をするようになっていた影響も大きかった。Aさんが深夜や休日も延々と働く姿を見て、目を覚ますよう何度も説得したという
。 最後に、Aさんは筆者が代表を務めるNPO法人POSSE、そして個人加盟の労働組合のブラック企業ユニオンに辿り着き、ワタミの労働問題を大々的に告発することを決
意した。こうして、周囲の人たちに恵まれ、支援者に出会えたことで、Aさんはワタミの「洗脳」を脱し、これまでの労働問題を直視できるようになったのだ。 Aさんはいま、
自分がワタミの労働問題に加担したのではという自責の念を抱いている。「ひとつ間違えば、私も上司と同じことをやったと思います」。自分の責任を果たすべく、これからもA
さんは、ワタミの告発を続けていくつもりだという。この記事の写真(8枚)
    \end{Verbatim}


    % Add a bibliography block to the postdoc
    
    
    
\end{document}
